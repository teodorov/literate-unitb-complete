\documentclass{article}
\usepackage{geometry}
\usepackage{amsmath}
\usepackage{bsymb}
\usepackage{../unitb}
% \usepackage{eventb} 
\usepackage{calculational}

\begin{document}
	
\begin{machine}{m0}
	
\[
\variable{ b : \Bool }
\]
\begin{description}
\comment{b}{ indicates termination of the process }
\end{description}
\newevent{term}{terminate}

\begin{align}
\initialization{in0}{b = \false} \\
\progress{prog0}{\true}{b}
\end{align}

\begin{align}
\refine{prog0}{ensure}{term}{} \\
% \removecoarse{term}{default} % \weakento{term}{default}{sch0} \\
\cschedule{term}{sch0}{\true} 
\end{align}

\begin{align}
\evbcmeq{term}{act0}{b}{\true}
\end{align}
%!TEX root=../puzzle.tex
\begin{block}
  \item   \textbf{machine} m0
  \item   \textbf{variables}
  \begin{block}
    \item   $b$
    \item   \begin{block}
      \item    indicates termination of the process 
    \end{block}
  \end{block}
  \item   %!TEX root=../train-station-set.tex
\textbf{progress}
\begin{block}
\item[ \eqref{m0:prog0} ]{$t \in in  \quad \mapsto\quad \neg t \in in $} %
\end{block}

  \item   \textbf{events}
  \begin{block}
    \item   %!TEX root=../puzzle.tex
\noindent \ref{term}  \textbf{event}
\begin{block}
  \item   \textbf{during}
  \begin{block}
  \item[ (\ref{term}/default) ]\sout{$\false$} %
  \end{block}
  \begin{block}
  \item[ \eqref{termsch0} ]{$\true$} %
  \end{block}
  \item   \textbf{begin}
  \begin{block}
  \item[ \eqref{termact0} ]{$b \bcmeq \true$} %
  \end{block}
  \item   \textbf{end} \\
\end{block}

  \end{block}
  \item   \textbf{end} \\
\end{block}


\end{machine}

\newcommand{\Pcs}{\text{P}}

\begin{machine}{m1}
	\refines{m0}

\with{sets}
\begin{align*}	
\newset{\Pcs} \\
\variable{ vs : \set[\Pcs] } \\
\dummy{V : \set[\Pcs]}
\end{align*}\begin{description}
\comment{vs}{set of visited processes}\end{description}
\begin{align}
	\invariant{inv0}{b \2\implies& vs = \Pcs} \\
	\evguard{term}{grd0}{ vs &= \Pcs } \\
	\cschedule{term}{sch1}{ vs &= \Pcs }
\end{align}\begin{description}
\comment{\ref{inv0}}{ termination is characterized by everyone 
	having visited }\end{description}
\replace{term}{sch1}{prog1}{saf1}
\begin{align*}
	\safety{saf1}{vs = \Pcs}{\false}
\end{align*}
	
\begin{align*}
	&\progress{prog1}{\true}{vs = \Pcs} 
\refine{prog1}{induction}{prog2}
		{ \var{\Pcs \setminus vs} }
	&\progress{prog2}
		{ \Pcs \setminus vs = V }
		{ (\Pcs \setminus vs \subset V) \1\lor vs = \Pcs }
\refine{prog2}{psp}{prog3,saf2}{}
	&\progress{prog3}
		{ \Pcs \setminus vs = V \land \neg vs = \Pcs}
		{\neg \Pcs \setminus vs = V} \\
	&\safety{saf2}{\Pcs \setminus vs \subseteq V}{vs = \Pcs}
\refine{prog3}{ensure}{visit}{ [\index{p}{\neg p' \in vs}] }
\end{align*}

\begin{align*}
	\assumption{asm0}{ \finite.\Pcs }
\end{align*}

\newevent{visit}{visit}

\[ \indices{visit}{p : \Pcs} \]

\begin{align}
% 	\cschedule{visit}{sch0}{}
	\evbcmeq{visit}{act1}{vs}{ vs \bunion \{ p \} }
\end{align}

% \removecoarse{visit}{default} % \weakento{visit}{default}{}

% \input{puzzle/}
%!TEX root=../main8.tex
\begin{block}
  \item   \textbf{machine} m1
  \item   \textbf{variables}
  \begin{block}
    \item   $emp$
    \item   $p$
    \item   $ppL$
    \item   $ppR$
    \item   $psL$
    \item   $psR$
    \item   $q$
    \item   $qe$
    \item   $res$
  \end{block}
  \item   %!TEX root=../main.tex
\textbf{invariants}
\begin{block}
\item[ \eqref{m1:inv0} ]{$\between{0}{c}{v} $} %
\item[ \eqref{m1:inv1} ]{$c > 0 \land in.c \in \dom.parse  %
            \2\implies ast = parse.(in.c) $} %
\item[ \eqref{m1:inv2} ]{$err \2\equiv c > 0  %
            \1\land \neg in.c \in \dom.parse $} %
\item[ \eqref{m1:inv3} ]{$\between{c}{nx}{v} $} %
\end{block}

  \item   \textbf{events}
  \begin{block}
    \item   %!TEX root=../main8.tex
\noindent \ref{req:pop:left}  \textbf{event}
\begin{block}
  \item   \textbf{during}
  \begin{block}
  \item[ (\ref{req:pop:left}/default) ]{$\false $} %
  \end{block}
  \item   \textbf{any} r
  \item   \textbf{when}
  \begin{block}
  \item[ \eqref{req:pop:leftm0:grd0} ]{$\neg r \in ppL $} %
  \end{block}
  \item   \textbf{begin}
  \begin{block}
  \item[ \eqref{req:pop:leftm0:act0} ]{$ppL \bcmeq ppL \1\bunion \{r\} $} %
  \end{block}
  \item   \textbf{end} \\
\end{block}

    \item   %!TEX root=../main8.tex
\noindent \ref{req:pop:right}  \textbf{event}
\begin{block}
  \item   \textbf{during}
  \begin{block}
  \item[ (\ref{req:pop:right}/default) ]{$\false $} %
  \end{block}
  \item   \textbf{any} r
  \item   \textbf{when}
  \begin{block}
  \item[ \eqref{req:pop:rightm0:grd0} ]{$\neg r \in ppR $} %
  \end{block}
  \item   \textbf{begin}
  \begin{block}
  \item[ \eqref{req:pop:rightm0:act0} ]{$ppL \bcmeq ppL \1\bunion \{r\} $} %
  \end{block}
  \item   \textbf{end} \\
\end{block}

    \item   %!TEX root=../main8.tex
\noindent \ref{req:push:left}  \textbf{event}
\begin{block}
  \item   \textbf{during}
  \begin{block}
  \item[ (\ref{req:push:left}/default) ]{$\false $} %
  \end{block}
  \item   \textbf{any} r,x
  \item   \textbf{when}
  \begin{block}
  \item[ \eqref{req:push:leftm0:grd0} ]{$\neg r \in \dom.psL $} %
  \end{block}
  \item   \textbf{begin}
  \begin{block}
  \item[ \eqref{req:push:leftm0:act0} ]{$psL \bcmeq psL \1| r \fun x $} %
  \end{block}
  \item   \textbf{end} \\
\end{block}

    \item   %!TEX root=../main8.tex
\noindent \ref{req:push:right}  \textbf{event}
\begin{block}
  \item   \textbf{during}
  \begin{block}
  \item[ (\ref{req:push:right}/default) ]{$\false $} %
  \end{block}
  \item   \textbf{any} r,x
  \item   \textbf{when}
  \begin{block}
  \item[ \eqref{req:push:rightm0:grd0} ]{$\neg r \in \dom.psR $} %
  \end{block}
  \item   \textbf{begin}
  \begin{block}
  \item[ \eqref{req:push:rightm0:act0} ]{$psR \bcmeq psR \1| r \fun x $} %
  \end{block}
  \item   \textbf{end} \\
\end{block}

    \item   %!TEX root=../main8.tex
\noindent \ref{resp:pop:left} [r] \textbf{event}
\begin{block}
  \item   \textbf{during}
  \begin{block}
  \item[ \eqref{resp:pop:leftm0:sch0} ]{$r \in ppL $} %
  \end{block}
  \item   \textbf{upon}
  \begin{block}
  \item[ \eqref{resp:pop:leftm1:sch0} ]{$\neg p = q$} %
  \end{block}
  \item   \textbf{when}
  \begin{block}
  \item[ \eqref{resp:pop:leftm1:grd0} ]{$\neg p = q$} %
  \end{block}
  \item   \textbf{begin}
  \begin{block}
  \item[ \eqref{resp:pop:leftm0:act0} ]{$ppL \bcmeq ppL \setminus \{ r \} $} %
  \item[ \eqref{resp:pop:leftm1:act0} ]{$p \bcmeq p+1$} %
  \item[ \eqref{resp:pop:leftm1:act1} ]{$qe \bcmeq \{ p \} \domsub qe $} %
  \item[ \eqref{resp:pop:leftm1:act2} ]{$emp \bcmeq \false $} %
  \item[ \eqref{resp:pop:leftm1:act3} ]{$res \bcmeq qe.p $} %
  \end{block}
  \item   \textbf{end} \\
\end{block}

    \item   %!TEX root=../main8.tex
\noindent \ref{resp:pop:left:empty} [r] \textbf{event}
\begin{block}
  \item   \textbf{during}
  \begin{block}
  \item[ \eqref{resp:pop:left:emptym0:sch0} ]{$r \in ppL $} %
  \end{block}
  \item   \textbf{begin}
  \begin{block}
  \item[ \eqref{resp:pop:left:emptym0:act0} ]{$ppL \bcmeq ppL \setminus \{ r \} $} %
  \end{block}
  \item   \textbf{end} \\
\end{block}
  \item   \textbf{upon}
\begin{block}
  \begin{block}
  \item[ \eqref{resp:pop:left:emptym1:sch0} ]{$p = q $} %
  \end{block}
  \item   \textbf{when}
  \begin{block}
  \item[ \eqref{resp:pop:left:emptym1:grd0} ]{$p = q $} %
  \end{block}
  \item   \textbf{begin}
  \begin{block}
  \item[ \eqref{resp:pop:left:emptym0:act0} ]{$ppL \bcmeq ppL \setminus \{ r \} $} %
  \item[ \eqref{resp:pop:left:emptym1:act2} ]{$emp \bcmeq \true $} %
  \end{block}
  \item   \textbf{end} \\
\end{block}

    \item   %!TEX root=../main8.tex
\noindent \ref{resp:pop:right} [r] \textbf{event}
\begin{block}
  \item   \textbf{during}
  \begin{block}
  \item[ \eqref{resp:pop:rightm0:sch0} ]{$r \in ppR $} %
  \end{block}
  \item   \textbf{upon}
  \begin{block}
  \item[ \eqref{resp:pop:rightm1:sch0} ]{$\neg p = q$} %
  \end{block}
  \item   \textbf{when}
  \begin{block}
  \item[ \eqref{resp:pop:rightm1:grd0} ]{$\neg p = q$} %
  \end{block}
  \item   \textbf{begin}
  \begin{block}
  \item[ \eqref{resp:pop:rightm0:act0} ]{$ppR \bcmeq ppR \setminus \{ r \} $} %
  \item[ \eqref{resp:pop:rightm1:act0} ]{$q \bcmeq q-1$} %
  \item[ \eqref{resp:pop:rightm1:act1} ]{$qe \bcmeq \{ q\0-1 \} \domsub qe $} %
  \item[ \eqref{resp:pop:rightm1:act2} ]{$emp \bcmeq \false $} %
  \item[ \eqref{resp:pop:rightm1:act3} ]{$res \bcmeq qe.(q\0-1) $} %
  \end{block}
  \item   \textbf{end} \\
\end{block}

    \item   %!TEX root=../main8.tex
\noindent \ref{resp:pop:right:empty} [r] \textbf{event}
\begin{block}
  \item   \textbf{during}
  \begin{block}
  \item[ \eqref{resp:pop:right:emptym0:sch0} ]{$r \in ppR $} %
  \end{block}
  \item   \textbf{upon}
  \begin{block}
  \item[ \eqref{resp:pop:right:emptym1:sch0} ]{$p = q $} %
  \end{block}
  \item   \textbf{when}
  \begin{block}
  \item[ \eqref{resp:pop:right:emptym1:grd0} ]{$p = q $} %
  \end{block}
  \item   \textbf{begin}
  \begin{block}
  \item[ \eqref{resp:pop:right:emptym0:act0} ]{$ppR \bcmeq ppR \setminus \{ r \} $} %
  \item[ \eqref{resp:pop:right:emptym1:act2} ]{$emp \bcmeq \true $} %
  \end{block}
  \item   \textbf{end} \\
\end{block}

    \item   %!TEX root=../main8.tex
\noindent \ref{resp:push:left} [r] \textbf{event}
\begin{block}
  \item   \textbf{during}
  \begin{block}
  \item[ \eqref{resp:push:leftm0:sch0} ]{$r \in \dom.psL $} %
  \end{block}
  \item   \textbf{when}
  \begin{block}
  \item[ \eqref{resp:push:leftm1:grd0} ]{$r \in \dom.psL $} %
  \end{block}
  \item   \textbf{begin}
  \begin{block}
  \item[ \eqref{resp:push:leftm0:act0} ]{$psL \bcmeq \{r\} \domsub psL $} %
  \item[ \eqref{resp:push:leftm1:act0} ]{$p \bcmeq p-1$} %
  \item[ \eqref{resp:push:leftm1:act1} ]{$qe \bcmeq qe \1| p\0-1 \fun psL.r$} %
  \end{block}
  \item   \textbf{end} \\
\end{block}

    \item   %!TEX root=../main8.tex
\noindent \ref{resp:push:right} [r] \textbf{event}
\begin{block}
  \item   \textbf{during}
  \begin{block}
  \item[ \eqref{resp:push:rightm0:sch0} ]{$r \in \dom.psR $} %
  \end{block}
  \item   \textbf{when}
  \begin{block}
  \item[ \eqref{resp:push:rightm1:grd0} ]{$r \in \dom.psR $} %
  \end{block}
  \item   \textbf{begin}
  \begin{block}
  \item[ \eqref{resp:push:rightm0:act0} ]{$psR \bcmeq \{r\} \domsub psR $} %
  \item[ \eqref{resp:push:rightm1:act0} ]{$q \bcmeq q+1$} %
  \item[ \eqref{resp:push:rightm1:act1} ]{$qe \bcmeq qe \1| q \fun psR.r $} %
  \end{block}
  \item   \textbf{end} \\
\end{block}

  \end{block}
  \item   \textbf{end} \\
\end{block}


\end{machine}

\begin{machine}{m2}
\refines{m1}

\[ \variable{ts : \set [\Pcs] } \]\begin{description}
\comment{ts}{set of process detected to have visited} \end{description}
\begin{align}
	\cschedule{term}{sch2}{ ts = \Pcs }
\end{align}

\replace{term}{sch2}{prog4}{saf3}

\begin{align}
	\safety{saf3}{ ts = \Pcs }{ \neg vs = \Pcs } \\
	\progress{prog4}{ \true }{ ts = \Pcs } \\
	\invariant{inv1}{ ts \subseteq vs } \\
	\initialization{in1}{ ts = \emptyset }
\end{align}

\[ \variable{ cs : \set [\Pcs] } \]\begin{description}
\comment{cs}{one-slot channel used to notify the counter of visitations} \end{description}
\begin{align*}
	\refine{prog4}{transitivity}{prog10,prog9}{}
& \progress{prog10}{ \true }{ cs = \emptyset } \\
& \progress{prog9}{ cs = \emptyset }{ ts = \Pcs }
	\refine{prog9}{induction}{prog5}{ \var{\Pcs \setminus ts} }
& \progress{prog5}
	 	{ cs = \emptyset \land ts = V }
	 	{ (cs = \emptyset \land V \subset ts) \lor ts = \Pcs }
	\refine{prog5}{PSP}{prog8,saf8}{}
& \progress{prog8}
		{ cs = \emptyset \land ts = V \land \neg ts = \Pcs }
		{ cs = \emptyset \land \neg ts = V }
\\& \safety{saf8}{ V \subseteq ts }{ ts = \Pcs }
	\refine{prog8}{transitivity}{prog11,prog6,prog7}{}
& \progress{prog11}
		{ cs = \emptyset \land ts = V \land \neg ts = \Pcs }
		{ cs = \emptyset \land ts = V \land \neg ts = \Pcs 
			\land \neg ts = vs } \\
& \progress{prog6}
		{ cs = \emptyset \land ts = V \land \neg ts = \Pcs \land \neg ts = vs }
		{ ts = V \land \neg cs = \emptyset \land cs \subseteq \Pcs \setminus ts } \\
& \progress{prog7}
		{ ts = V \land \neg cs = \emptyset 
				 \land cs \subseteq \Pcs \setminus ts }
		{ cs = \emptyset \land \neg ts = V }
	\refine{prog6}{ensure}{flick}{ \index{p}{p' \in vs \setminus ts} }
	\refine{prog7}{ensure}{count}{} % \index{p}{ p' \in cs } }
	\refine{prog11}{ensure}{visit}{ \index{p}{ \neg p' \in vs} }
\end{align*}

\begin{align*}
	% \refine{prog10}{trading}{prog11}{}
% & \progress{prog11}{ \neg cs = \emptyset }{ cs = \emptyset }
	\refine{prog10}{ensure}{count}{} % \index{p}{ p' \in cs } }
\end{align*}

\newevent{flick}{flick}
\newevent{count}{count}

\[ \indices{flick}{p : \Pcs} \]

\begin{align}
	\cschedule{flick}{sch0}{ cs = \emptyset } \\
	\cschedule{flick}{sch1}{ \neg p \in ts } \\
	\evguard{flick}{grd0}{ \neg p \in ts } \\
	\evbcmeq{flick}{act0}{cs}{ \{ p \} }
\end{align}

% \removecoarse{flick}{default} % \weakento{flick}{default}{sch0,sch1,sch2}

%!TEX root=../puzzle.tex
\noindent \ref{term}  \textbf{event}
\begin{block}
  \item   \textbf{during}
  \begin{block}
  \item[ \eqref{termsch1} ]\sout{$vs = \Pcs$} %
  \end{block}
  \begin{block}
  \item[ \eqref{termsch2} ]{$ts = \Pcs $} %
  \end{block}
  \item   \textbf{when}
  \begin{block}
  \item[ \eqref{termgrd0} ]{$vs = \Pcs$} %
  \end{block}
  \item   \textbf{begin}
  \begin{block}
  \item[ \eqref{termact0} ]{$b \bcmeq \true$} %
  \end{block}
  \item   \textbf{end} \\
\end{block}

%!TEX root=../puzzle.tex
\noindent \ref{flick} [p] \textbf{event}
\begin{block}
  \item   \textbf{during}
  \begin{block}
  \item[ (\ref{flick}/default) ]\sout{$\false$} %
  \end{block}
  \begin{block}
  \item[ \eqref{flicksch0} ]{$cs = \emptyset $} %
  \item[ \eqref{flicksch1} ]{$\neg p \in ts $} %
  \item[ \eqref{flicksch2} ]{$p \in vs$} %
  \end{block}
  \item   \textbf{when}
  \begin{block}
  \item[ \eqref{flickgrd0} ]{$\neg p \in ts $} %
  \item[ \eqref{flickgrd1} ]{$p \in vs$} %
  \end{block}
  \item   \textbf{begin}
  \begin{block}
  \item[ \eqref{flickact0} ]{$cs \bcmeq \{ p \} $} %
  \end{block}
  \item   \textbf{end} \\
\end{block}


% \[ \indices{count}{p : \Pcs} \]

\begin{align}
	\invariant{inv2}{cs \subseteq vs} \\
	\initialization{in2}{ cs = \emptyset } \\
	\cschedule{count}{sch0}{ \neg cs = \emptyset } \\
	% \cschedule{count}{sch1}{ cs \subseteq vs } \\
	% \evguard{count}{grd0}{ cs \subseteq vs } \\
	\evbcmeq{count}{act0}{cs}{\emptyset} \\
	\evbcmeq{count}{act1}{ts}{ ts \bunion cs }
\end{align}

% \removecoarse{count}{default} % \weakento{count}{default}{sch0}

%!TEX root=../puzzle.tex
\noindent \ref{count}  \textbf{event}
\begin{block}
  \item   \textbf{during}
  \begin{block}
  \item[ (\ref{count}/default) ]\sout{$\false$} %
  \end{block}
  \begin{block}
  \item[ \eqref{countsch0} ]{$\neg cs = \emptyset $} %
  \end{block}
  \item   \textbf{begin}
  \begin{block}
  \item[ \eqref{countact0} ]{$cs \bcmeq \emptyset$} %
  \item[ \eqref{countact1} ]{$ts \bcmeq ts \bunion cs $} %
  \end{block}
  \item   \textbf{end} \\
\end{block}


\begin{align*}
	\evguard{flick}{grd1}{p \in vs} \\
	\cschedule{flick}{sch2}{p \in vs}
\end{align*}

%!TEX root=../main9.tex
\begin{block}
  \item   \textbf{machine} m2
  \item   \textbf{refines} m0
  \item   \textbf{variables}
  \begin{block}
    \item   $req$
    \item   $req0$
    \item   $reqA$
    \item   $reqB$
  \end{block}
  \item   %!TEX root=../main.tex
\textbf{invariants}
\begin{block}
\item[ \eqref{m2:inv0} ]{$r \in reach $} %
\end{block}

  \item   \textbf{events}
  \begin{block}
    \item   %!TEX root=../main9.tex
\noindent \ref{handle}  \textbf{event}
\begin{block}
  \item   \textbf{during}
  \begin{block}
  \item[ \eqref{handlem0:sch0} ]{$\neg req = \emptyset$} %
  \end{block}
  \item   \textbf{any} r
  \item   \textbf{when}
  \begin{block}
  \item[ \eqref{handlegrd0} ]{$r \in req$} %
  \end{block}
  \item   \textbf{begin}
  \begin{block}
  \item[ \eqref{handleact0} ]{$req \bcmeq req \setminus \{ r \}$} %
  \item[ \eqref{handleact1} ]{$req0 \bcmeq req$} %
  \end{block}
  \item   \textbf{end} \\
\end{block}

    \item   %!TEX root=../main9.tex
\noindent \ref{req}  \textbf{event}
\begin{block}
  \item   \textbf{during}
  \begin{block}
  \item[ (\ref{req}/default) ]{$\false$} %
  \end{block}
  \item   \textbf{any} r
  \item   \textbf{when}
  \begin{block}
  \item[ \eqref{reqgrd0} ]{$\neg r \in req$} %
  \end{block}
  \item   \textbf{begin}
  \begin{block}
  \item[ \eqref{reqact0} ]{$req \bcmeq req \bunion \{ r \}$} %
  \item[ \eqref{reqact1} ]{$req0 \bcmeq req$} %
  \end{block}
  \item   \textbf{end} \\
\end{block}

  \end{block}
  \item   \textbf{end} \\
\end{block}


\end{machine}
\begin{machine}{m3}
	\refines{m2}
	\[ \variable{ c : \Int } \]
	\begin{description}
\comment{flick}{stuff}
		\comment{c}{ one-bit channel used to communicate }
	\end{description}\removevar{cs}
	\removevar{ts}
	\removeact{count}{act0,act1}\witness{count}{cs}{cs' = \emptyset}
	\removeact{flick}{act}
	\removeinit{in} \removeinit{in1}
	\initwitness{cs}{cs = \emptyset}
	\initwitness{ts}{ts = \emptyset}
	\begin{align} 
		\invariant{m3:inv5}{ c \in \{0,1\} } \\
		\invariant{m3:inv0}{ c = \card.cs } \\
		\initialization{m3:in0}{ c = 0 } \\ 
		\evbcmeq{count}{m3:act0}{c}{0} \\
		\evbcmeq{flick}{m3:act0}{c}{1}
	\end{align}
	\[ \variable{ n : \Int } \]
	\begin{align}
		\invariant{m3:inv1}{ n = \qsum{p}{p \in ts}{1} } \\
		\invariant{m3:inv2}{ \finite.ts }\\
		\invariant{m3:inv3}{ \finite.cs } \\
		\initialization{m3:in1}{ n = 0 }  \\
		% \invariant{m3:inv4}{ \qsum{p}{p \in ts \bunion cs}{1} = n + c }\\
		\invariant{m3:inv6}{ ts \binter cs = \emptyset } \\
		\evbcmeq{count}{m3:act1}{n}{n+1} \\
		\evguard{count}{m3:grd0}{ c = 1 }
	\end{align}
	%!TEX root=../puzzle.tex
\begin{block}
  \item   \textbf{machine} m3
  \item   \textbf{variables}
  \begin{block}
    \item   $cs$\quad(removed)
    \item   \begin{block}
      \item   one-slot channel used to notify the counter 
    \end{block}
    \item   $ts$\quad(removed)
    \item   \begin{block}
      \item   set of process detected to have visited
    \end{block}
    \item   $b$
    \item   \begin{block}
      \item    indicates termination of the process 
    \end{block}
    \item   $c$
    \item   \begin{block}
      \item    one-bit channel used to communicate 
    \end{block}
    \item   $fs$
    \item   \begin{block}
      \item    $p \in fs$ if process $p$ has flicked the switch 
    \end{block}
    \item   $n$
    \item   $vs$
    \item   \begin{block}
      \item   set of visited processes
    \end{block}
  \end{block}
  \item   %!TEX root=../puzzle.tex
\textbf{invariants}
\begin{block}
\item[ \eqref{m3:inv0} ]{$c = \card.cs $} %
\item[ \eqref{m3:inv1} ]{$n = \qsum{p}{p \in ts}{1} $} %
\item[ \eqref{m3:inv2} ]{$\finite.ts $} %
\item[ \eqref{m3:inv5} ]{$c \in \{0,1\} $} %
\item[ \eqref{m3:inv6} ]{$ts \binter cs = \emptyset $} %
\item[ \eqref{m3:inv7} ]{$fs = ts \bunion cs $} %
\end{block}

  \item   %!TEX root=../train-station-set.tex

  \item   \textbf{events}
  \begin{block}
    \item   %!TEX root=../puzzle.tex
\noindent \ref{count}  \textbf{event}
\begin{block}
  \item   \textbf{during}
  \begin{block}
  \item[ \eqref{countsch0} ]\sout{$\neg cs = \emptyset $} %
  \end{block}
  \begin{block}
  \item[ \eqref{countsch1} ]{$c = 1$} %
  \end{block}
  \item   \textbf{when}
  \begin{block}
  \item[ \eqref{countm3:grd0} ]{$c = 1$} %
  \end{block}
  \item   \textbf{begin}
  \begin{block}
  \item[ \eqref{countact0} ]\sout{$cs \bcmeq \emptyset$} %
  \item[ \eqref{countact1} ]\sout{$ts \bcmeq ts \bunion cs $} %
  \end{block}
  \begin{block}
  \item[ \eqref{countm3:act0} ]{$c \bcmeq 0$} %
  \item[ \eqref{countm3:act1} ]{$n \bcmeq n+1$} %
  \end{block}
  \item   \textbf{end} \\
\end{block}

    \item   \noindent \ref{flick} [p] \textbf{event}
  \item   \begin{block}
    \item   stuff
  \end{block}
\begin{block}
  \item   \textbf{during}
  \begin{block}
  \item[ \eqref{flickm3:csch1} ]$c = 0 $ %
  \item[ \eqref{flickm3:csch2} ]$\neg p \in fs $ %
  \item[ \eqref{flicksch2} ]$p \in vs$ %
  \end{block}
  \item   \textbf{when}
  \begin{block}
  \item[ \eqref{flickgrd1} ]$p \in vs$ %
  \item[ \eqref{flickm3:grd1} ]$c = 0 $ %
  \item[ \eqref{flickm3:grd2} ]$\neg p \in fs $ %
  \end{block}
  \item   \textbf{begin}
  \begin{block}
  \item[ \eqref{flickm3:act0} ]$c \bcmeq 1$ %
  \item[ \eqref{flickm3:act2} ]$fs \bcmeq fs \bunion \{ p \} $ %
  \end{block}
  \item   \textbf{end} \\
\end{block}

    \item   %!TEX root=../puzzle.tex
\noindent \ref{term}  \textbf{event}
\begin{block}
  \item   \textbf{during}
  \begin{block}
  \item[ \eqref{termsch2} ]\sout{$ts = \Pcs $} %
  \end{block}
  \begin{block}
  \item[ \eqref{termsch3} ]{$n = \card.\Pcs$} %
  \end{block}
  \item   \textbf{when}
  \begin{block}
  \item[ \eqref{termgrd0} ]{$vs = \Pcs$} %
  \end{block}
  \item   \textbf{begin}
  \begin{block}
  \item[ \eqref{termact0} ]{$b \bcmeq \true$} %
  \end{block}
  \item   \textbf{end} \\
\end{block}

    \item   %!TEX root=../puzzle.tex
\noindent \ref{visit} [p] \textbf{event}
\begin{block}
  \item   \textbf{begin}
  \begin{block}
  \item[ \eqref{visitact1} ]{$vs \bcmeq vs \bunion \{ p \} $} %
  \end{block}
  \item   \textbf{end} \\
\end{block}

  \end{block}
  \item   \textbf{end} \\
\end{block}

\end{machine}

\begin{machine}{m4}
	\refines{m3}
	% \invariant{m4:inv0}{cs \subseteq cs}
\end{machine}

% \end{document}
%!TEX root=../puzzle.tex
\begin{block}
  \item   \textbf{machine} m0
  \item   \textbf{variables}
  \begin{block}
    \item   $b$
    \item   \begin{block}
      \item    indicates termination of the process 
    \end{block}
  \end{block}
  \item   %!TEX root=../train-station-set.tex
\textbf{progress}
\begin{block}
\item[ \eqref{m0:prog0} ]{$t \in in  \quad \mapsto\quad \neg t \in in $} %
\end{block}

  \item   \textbf{events}
  \begin{block}
    \item   %!TEX root=../puzzle.tex
\noindent \ref{term}  \textbf{event}
\begin{block}
  \item   \textbf{during}
  \begin{block}
  \item[ (\ref{term}/default) ]\sout{$\false$} %
  \end{block}
  \begin{block}
  \item[ \eqref{termsch0} ]{$\true$} %
  \end{block}
  \item   \textbf{begin}
  \begin{block}
  \item[ \eqref{termact0} ]{$b \bcmeq \true$} %
  \end{block}
  \item   \textbf{end} \\
\end{block}

  \end{block}
  \item   \textbf{end} \\
\end{block}


% \end{machine}

\newcommand{\Pcs}{\text{P}}

\begin{machine}{m1}
	% \refines{m0}

% \with{sets}
\begin{align*}	
% \newset{\Pcs} \\
% \variable{ vs : \set[\Pcs] } \\
% \dummy{V : \set[\Pcs]}
\end{align*}\begin{description}
\comment{vs}{set of visited processes}\end{description}
\begin{align}
	% \invariant{inv0}{b \2\implies& vs = \Pcs} \\
	\evguard{term}{grd0}{ vs &= \Pcs } \\
	\cschedule{term}{sch1}{ vs &= \Pcs }
\end{align}\begin{description}
\comment{\ref{inv0}}{ termination is characterized by everyone 
	having visited }\end{description}
\replace{term}{sch1}{prog1}{saf1}
\begin{align*}
	% \safety{saf1}{vs = \Pcs}{\false}
\end{align*}
	
\begin{align*}
	% &\progress{prog1}{\true}{vs = \Pcs} 
\refine{prog1}{induction}{prog2}
		{ \var{\Pcs \setminus vs} }
	% &\progress{prog2}
		{ \Pcs \setminus vs = V }
		{ (\Pcs \setminus vs \subset V) \1\lor vs = \Pcs }
\refine{prog2}{psp}{prog3,saf2}{}
	% &\progress{prog3}
		{ \Pcs \setminus vs = V \land \neg vs = \Pcs}
		{\neg \Pcs \setminus vs = V} \\
	% &\safety{saf2}{\Pcs \setminus vs \subseteq V}{vs = \Pcs}
\refine{prog3}{ensure}{visit}{ [\index{p}{\neg p' \in vs}] }
\end{align*}

\begin{align*}
	% \assumption{asm0}{ \finite.\Pcs }
\end{align*}

% \newevent{visit}{visit}

% \[ \indices{visit}{p : \Pcs} \]

\begin{align}
% 	\cschedule{visit}{sch0}{}
	\evbcmeq{visit}{act1}{vs}{ vs \bunion \{ p \} }
\end{align}

% \removecoarse{visit}{default} % \weakento{visit}{default}{}

% \input{}
%!TEX root=../main8.tex
\begin{block}
  \item   \textbf{machine} m1
  \item   \textbf{variables}
  \begin{block}
    \item   $emp$
    \item   $p$
    \item   $ppL$
    \item   $ppR$
    \item   $psL$
    \item   $psR$
    \item   $q$
    \item   $qe$
    \item   $res$
  \end{block}
  \item   %!TEX root=../main.tex
\textbf{invariants}
\begin{block}
\item[ \eqref{m1:inv0} ]{$\between{0}{c}{v} $} %
\item[ \eqref{m1:inv1} ]{$c > 0 \land in.c \in \dom.parse  %
            \2\implies ast = parse.(in.c) $} %
\item[ \eqref{m1:inv2} ]{$err \2\equiv c > 0  %
            \1\land \neg in.c \in \dom.parse $} %
\item[ \eqref{m1:inv3} ]{$\between{c}{nx}{v} $} %
\end{block}

  \item   \textbf{events}
  \begin{block}
    \item   %!TEX root=../main8.tex
\noindent \ref{req:pop:left}  \textbf{event}
\begin{block}
  \item   \textbf{during}
  \begin{block}
  \item[ (\ref{req:pop:left}/default) ]{$\false $} %
  \end{block}
  \item   \textbf{any} r
  \item   \textbf{when}
  \begin{block}
  \item[ \eqref{req:pop:leftm0:grd0} ]{$\neg r \in ppL $} %
  \end{block}
  \item   \textbf{begin}
  \begin{block}
  \item[ \eqref{req:pop:leftm0:act0} ]{$ppL \bcmeq ppL \1\bunion \{r\} $} %
  \end{block}
  \item   \textbf{end} \\
\end{block}

    \item   %!TEX root=../main8.tex
\noindent \ref{req:pop:right}  \textbf{event}
\begin{block}
  \item   \textbf{during}
  \begin{block}
  \item[ (\ref{req:pop:right}/default) ]{$\false $} %
  \end{block}
  \item   \textbf{any} r
  \item   \textbf{when}
  \begin{block}
  \item[ \eqref{req:pop:rightm0:grd0} ]{$\neg r \in ppR $} %
  \end{block}
  \item   \textbf{begin}
  \begin{block}
  \item[ \eqref{req:pop:rightm0:act0} ]{$ppL \bcmeq ppL \1\bunion \{r\} $} %
  \end{block}
  \item   \textbf{end} \\
\end{block}

    \item   %!TEX root=../main8.tex
\noindent \ref{req:push:left}  \textbf{event}
\begin{block}
  \item   \textbf{during}
  \begin{block}
  \item[ (\ref{req:push:left}/default) ]{$\false $} %
  \end{block}
  \item   \textbf{any} r,x
  \item   \textbf{when}
  \begin{block}
  \item[ \eqref{req:push:leftm0:grd0} ]{$\neg r \in \dom.psL $} %
  \end{block}
  \item   \textbf{begin}
  \begin{block}
  \item[ \eqref{req:push:leftm0:act0} ]{$psL \bcmeq psL \1| r \fun x $} %
  \end{block}
  \item   \textbf{end} \\
\end{block}

    \item   %!TEX root=../main8.tex
\noindent \ref{req:push:right}  \textbf{event}
\begin{block}
  \item   \textbf{during}
  \begin{block}
  \item[ (\ref{req:push:right}/default) ]{$\false $} %
  \end{block}
  \item   \textbf{any} r,x
  \item   \textbf{when}
  \begin{block}
  \item[ \eqref{req:push:rightm0:grd0} ]{$\neg r \in \dom.psR $} %
  \end{block}
  \item   \textbf{begin}
  \begin{block}
  \item[ \eqref{req:push:rightm0:act0} ]{$psR \bcmeq psR \1| r \fun x $} %
  \end{block}
  \item   \textbf{end} \\
\end{block}

    \item   %!TEX root=../main8.tex
\noindent \ref{resp:pop:left} [r] \textbf{event}
\begin{block}
  \item   \textbf{during}
  \begin{block}
  \item[ \eqref{resp:pop:leftm0:sch0} ]{$r \in ppL $} %
  \end{block}
  \item   \textbf{upon}
  \begin{block}
  \item[ \eqref{resp:pop:leftm1:sch0} ]{$\neg p = q$} %
  \end{block}
  \item   \textbf{when}
  \begin{block}
  \item[ \eqref{resp:pop:leftm1:grd0} ]{$\neg p = q$} %
  \end{block}
  \item   \textbf{begin}
  \begin{block}
  \item[ \eqref{resp:pop:leftm0:act0} ]{$ppL \bcmeq ppL \setminus \{ r \} $} %
  \item[ \eqref{resp:pop:leftm1:act0} ]{$p \bcmeq p+1$} %
  \item[ \eqref{resp:pop:leftm1:act1} ]{$qe \bcmeq \{ p \} \domsub qe $} %
  \item[ \eqref{resp:pop:leftm1:act2} ]{$emp \bcmeq \false $} %
  \item[ \eqref{resp:pop:leftm1:act3} ]{$res \bcmeq qe.p $} %
  \end{block}
  \item   \textbf{end} \\
\end{block}

    \item   %!TEX root=../main8.tex
\noindent \ref{resp:pop:left:empty} [r] \textbf{event}
\begin{block}
  \item   \textbf{during}
  \begin{block}
  \item[ \eqref{resp:pop:left:emptym0:sch0} ]{$r \in ppL $} %
  \end{block}
  \item   \textbf{begin}
  \begin{block}
  \item[ \eqref{resp:pop:left:emptym0:act0} ]{$ppL \bcmeq ppL \setminus \{ r \} $} %
  \end{block}
  \item   \textbf{end} \\
\end{block}
  \item   \textbf{upon}
\begin{block}
  \begin{block}
  \item[ \eqref{resp:pop:left:emptym1:sch0} ]{$p = q $} %
  \end{block}
  \item   \textbf{when}
  \begin{block}
  \item[ \eqref{resp:pop:left:emptym1:grd0} ]{$p = q $} %
  \end{block}
  \item   \textbf{begin}
  \begin{block}
  \item[ \eqref{resp:pop:left:emptym0:act0} ]{$ppL \bcmeq ppL \setminus \{ r \} $} %
  \item[ \eqref{resp:pop:left:emptym1:act2} ]{$emp \bcmeq \true $} %
  \end{block}
  \item   \textbf{end} \\
\end{block}

    \item   %!TEX root=../main8.tex
\noindent \ref{resp:pop:right} [r] \textbf{event}
\begin{block}
  \item   \textbf{during}
  \begin{block}
  \item[ \eqref{resp:pop:rightm0:sch0} ]{$r \in ppR $} %
  \end{block}
  \item   \textbf{upon}
  \begin{block}
  \item[ \eqref{resp:pop:rightm1:sch0} ]{$\neg p = q$} %
  \end{block}
  \item   \textbf{when}
  \begin{block}
  \item[ \eqref{resp:pop:rightm1:grd0} ]{$\neg p = q$} %
  \end{block}
  \item   \textbf{begin}
  \begin{block}
  \item[ \eqref{resp:pop:rightm0:act0} ]{$ppR \bcmeq ppR \setminus \{ r \} $} %
  \item[ \eqref{resp:pop:rightm1:act0} ]{$q \bcmeq q-1$} %
  \item[ \eqref{resp:pop:rightm1:act1} ]{$qe \bcmeq \{ q\0-1 \} \domsub qe $} %
  \item[ \eqref{resp:pop:rightm1:act2} ]{$emp \bcmeq \false $} %
  \item[ \eqref{resp:pop:rightm1:act3} ]{$res \bcmeq qe.(q\0-1) $} %
  \end{block}
  \item   \textbf{end} \\
\end{block}

    \item   %!TEX root=../main8.tex
\noindent \ref{resp:pop:right:empty} [r] \textbf{event}
\begin{block}
  \item   \textbf{during}
  \begin{block}
  \item[ \eqref{resp:pop:right:emptym0:sch0} ]{$r \in ppR $} %
  \end{block}
  \item   \textbf{upon}
  \begin{block}
  \item[ \eqref{resp:pop:right:emptym1:sch0} ]{$p = q $} %
  \end{block}
  \item   \textbf{when}
  \begin{block}
  \item[ \eqref{resp:pop:right:emptym1:grd0} ]{$p = q $} %
  \end{block}
  \item   \textbf{begin}
  \begin{block}
  \item[ \eqref{resp:pop:right:emptym0:act0} ]{$ppR \bcmeq ppR \setminus \{ r \} $} %
  \item[ \eqref{resp:pop:right:emptym1:act2} ]{$emp \bcmeq \true $} %
  \end{block}
  \item   \textbf{end} \\
\end{block}

    \item   %!TEX root=../main8.tex
\noindent \ref{resp:push:left} [r] \textbf{event}
\begin{block}
  \item   \textbf{during}
  \begin{block}
  \item[ \eqref{resp:push:leftm0:sch0} ]{$r \in \dom.psL $} %
  \end{block}
  \item   \textbf{when}
  \begin{block}
  \item[ \eqref{resp:push:leftm1:grd0} ]{$r \in \dom.psL $} %
  \end{block}
  \item   \textbf{begin}
  \begin{block}
  \item[ \eqref{resp:push:leftm0:act0} ]{$psL \bcmeq \{r\} \domsub psL $} %
  \item[ \eqref{resp:push:leftm1:act0} ]{$p \bcmeq p-1$} %
  \item[ \eqref{resp:push:leftm1:act1} ]{$qe \bcmeq qe \1| p\0-1 \fun psL.r$} %
  \end{block}
  \item   \textbf{end} \\
\end{block}

    \item   %!TEX root=../main8.tex
\noindent \ref{resp:push:right} [r] \textbf{event}
\begin{block}
  \item   \textbf{during}
  \begin{block}
  \item[ \eqref{resp:push:rightm0:sch0} ]{$r \in \dom.psR $} %
  \end{block}
  \item   \textbf{when}
  \begin{block}
  \item[ \eqref{resp:push:rightm1:grd0} ]{$r \in \dom.psR $} %
  \end{block}
  \item   \textbf{begin}
  \begin{block}
  \item[ \eqref{resp:push:rightm0:act0} ]{$psR \bcmeq \{r\} \domsub psR $} %
  \item[ \eqref{resp:push:rightm1:act0} ]{$q \bcmeq q+1$} %
  \item[ \eqref{resp:push:rightm1:act1} ]{$qe \bcmeq qe \1| q \fun psR.r $} %
  \end{block}
  \item   \textbf{end} \\
\end{block}

  \end{block}
  \item   \textbf{end} \\
\end{block}


\end{machine}

\begin{machine}{m2}
% \refines{m1}

% \[ \variable{ts : \set [\Pcs] } \]\begin{description}
% \comment{ts}{set of process detected to have visited} \end{description}
% \begin{align}
% 	\cschedule{term}{sch2}{ ts = \Pcs }
% \end{align}

\replace{term}{sch2}{prog4}{saf3}

\begin{align}
	% \safety{saf3}{ ts = \Pcs }{ \neg vs = \Pcs } \\
	% \progress{prog4}{ \true }{ ts = \Pcs } \\
	% \invariant{inv1}{ ts \subseteq vs } \\
	% \initialization{in1}{ ts = \emptyset }
\end{align}
% \[ \variable{ cs : \set [\Pcs] } \]
\begin{description}
\comment{cs}{one-slot channel used to notify the counter of visitations} \end{description}
\begin{align*}
	% \refine{prog4}{transitivity}{prog10,prog9}{}
% & \progress{prog10}{ \true }{ cs = \emptyset } \\
% & \progress{prog9}{ cs = \emptyset }{ ts = \Pcs }
	% \refine{prog9}{induction}{prog5}{ \var{\Pcs \setminus ts} }
% & \progress{prog5}
	 	{ cs = \emptyset \land ts = V }
	 	{ (cs = \emptyset \land V \subset ts) \lor ts = \Pcs }
	% \refine{prog5}{PSP}{prog8,saf8}{}
% & \progress{prog8}
		{ cs = \emptyset \land ts = V \land \neg ts = \Pcs }
		{ cs = \emptyset \land \neg ts = V }
% \\& \safety{saf8}{ V \subseteq ts }{ ts = \Pcs }
	% \refine{prog8}{transitivity}{prog11,prog6,prog7}{}
% & \progress{prog11}
		{ cs = \emptyset \land ts = V \land \neg ts = \Pcs }
		{ cs = \emptyset \land ts = V \land \neg ts = \Pcs 
			\land \neg ts = vs } \\
% & \progress{prog6}
		{ cs = \emptyset \land ts = V \land \neg ts = \Pcs \land \neg ts = vs }
		{ ts = V \land \neg cs = \emptyset \land cs \subseteq \Pcs \setminus ts } \\
% & \progress{prog7}
		{ ts = V \land \neg cs = \emptyset 
				 \land cs \subseteq \Pcs \setminus ts }
		{ cs = \emptyset \land \neg ts = V }
	\refine{prog6}{ensure}{flick}{ \index{p}{p' \in vs \setminus ts} }
	\refine{prog7}{ensure}{count}{} % \index{p}{ p' \in cs } }
	\refine{prog11}{ensure}{visit}{ \index{p}{ \neg p' \in vs} }
\end{align*}

\begin{align*}
	% \refine{prog10}{trading}{prog11}{}
% & \progress{prog11}{ \neg cs = \emptyset }{ cs = \emptyset }
	\refine{prog10}{ensure}{count}{} % \index{p}{ p' \in cs } }
\end{align*}

% \newevent{flick}{flick}
% \newevent{count}{count}

% \[ \indices{flick}{p : \Pcs} \]

\begin{align}
	\cschedule{flick}{sch0}{ cs = \emptyset } \\
	\cschedule{flick}{sch1}{ \neg p \in ts } \\
	\evguard{flick}{grd0}{ \neg p \in ts } \\
	\evbcmeq{flick}{act0}{cs}{ \{ p \} }
\end{align}

% \removecoarse{flick}{default} % \weakento{flick}{default}{sch0,sch1,sch2}

%!TEX root=../puzzle.tex
\noindent \ref{term}  \textbf{event}
\begin{block}
  \item   \textbf{during}
  \begin{block}
  \item[ \eqref{termsch1} ]\sout{$vs = \Pcs$} %
  \end{block}
  \begin{block}
  \item[ \eqref{termsch2} ]{$ts = \Pcs $} %
  \end{block}
  \item   \textbf{when}
  \begin{block}
  \item[ \eqref{termgrd0} ]{$vs = \Pcs$} %
  \end{block}
  \item   \textbf{begin}
  \begin{block}
  \item[ \eqref{termact0} ]{$b \bcmeq \true$} %
  \end{block}
  \item   \textbf{end} \\
\end{block}

%!TEX root=../puzzle.tex
\noindent \ref{flick} [p] \textbf{event}
\begin{block}
  \item   \textbf{during}
  \begin{block}
  \item[ (\ref{flick}/default) ]\sout{$\false$} %
  \end{block}
  \begin{block}
  \item[ \eqref{flicksch0} ]{$cs = \emptyset $} %
  \item[ \eqref{flicksch1} ]{$\neg p \in ts $} %
  \item[ \eqref{flicksch2} ]{$p \in vs$} %
  \end{block}
  \item   \textbf{when}
  \begin{block}
  \item[ \eqref{flickgrd0} ]{$\neg p \in ts $} %
  \item[ \eqref{flickgrd1} ]{$p \in vs$} %
  \end{block}
  \item   \textbf{begin}
  \begin{block}
  \item[ \eqref{flickact0} ]{$cs \bcmeq \{ p \} $} %
  \end{block}
  \item   \textbf{end} \\
\end{block}


% \[ \indices{count}{p : \Pcs} \]

\begin{align}
	% \invariant{inv2}{cs \subseteq vs} \\
	% \initialization{in2}{ cs = \emptyset } \\
	\cschedule{count}{sch0}{ \neg cs = \emptyset } \\
	% \cschedule{count}{sch1}{ cs \subseteq vs } \\
	% \evguard{count}{grd0}{ cs \subseteq vs } \\
	\evbcmeq{count}{act0}{cs}{\emptyset} \\
	\evbcmeq{count}{act1}{ts}{ ts \bunion cs }
\end{align}

% \removecoarse{count}{default} % \weakento{count}{default}{sch0}

%!TEX root=../puzzle.tex
\noindent \ref{count}  \textbf{event}
\begin{block}
  \item   \textbf{during}
  \begin{block}
  \item[ (\ref{count}/default) ]\sout{$\false$} %
  \end{block}
  \begin{block}
  \item[ \eqref{countsch0} ]{$\neg cs = \emptyset $} %
  \end{block}
  \item   \textbf{begin}
  \begin{block}
  \item[ \eqref{countact0} ]{$cs \bcmeq \emptyset$} %
  \item[ \eqref{countact1} ]{$ts \bcmeq ts \bunion cs $} %
  \end{block}
  \item   \textbf{end} \\
\end{block}


\begin{align*}
	\evguard{flick}{grd1}{p \in vs} \\
	\cschedule{flick}{sch2}{p \in vs}
\end{align*}

%!TEX root=../main9.tex
\begin{block}
  \item   \textbf{machine} m2
  \item   \textbf{refines} m0
  \item   \textbf{variables}
  \begin{block}
    \item   $req$
    \item   $req0$
    \item   $reqA$
    \item   $reqB$
  \end{block}
  \item   %!TEX root=../main.tex
\textbf{invariants}
\begin{block}
\item[ \eqref{m2:inv0} ]{$r \in reach $} %
\end{block}

  \item   \textbf{events}
  \begin{block}
    \item   %!TEX root=../main9.tex
\noindent \ref{handle}  \textbf{event}
\begin{block}
  \item   \textbf{during}
  \begin{block}
  \item[ \eqref{handlem0:sch0} ]{$\neg req = \emptyset$} %
  \end{block}
  \item   \textbf{any} r
  \item   \textbf{when}
  \begin{block}
  \item[ \eqref{handlegrd0} ]{$r \in req$} %
  \end{block}
  \item   \textbf{begin}
  \begin{block}
  \item[ \eqref{handleact0} ]{$req \bcmeq req \setminus \{ r \}$} %
  \item[ \eqref{handleact1} ]{$req0 \bcmeq req$} %
  \end{block}
  \item   \textbf{end} \\
\end{block}

    \item   %!TEX root=../main9.tex
\noindent \ref{req}  \textbf{event}
\begin{block}
  \item   \textbf{during}
  \begin{block}
  \item[ (\ref{req}/default) ]{$\false$} %
  \end{block}
  \item   \textbf{any} r
  \item   \textbf{when}
  \begin{block}
  \item[ \eqref{reqgrd0} ]{$\neg r \in req$} %
  \end{block}
  \item   \textbf{begin}
  \begin{block}
  \item[ \eqref{reqact0} ]{$req \bcmeq req \bunion \{ r \}$} %
  \item[ \eqref{reqact1} ]{$req0 \bcmeq req$} %
  \end{block}
  \item   \textbf{end} \\
\end{block}

  \end{block}
  \item   \textbf{end} \\
\end{block}


\end{machine}
\begin{machine}{m3}
	% \refines{m2}
	% \[ \variable{ c : \Int } \]
	\begin{description}
\comment{flick}{stuff}
		\comment{c}{ one-bit channel used to communicate }
	\end{description} %\removevar{cs}
	% \removevar{ts}
	% \removeact{count}{act0,act1} %\witness{count}{cs}{cs' = \emptyset}
	% \removeact{flick}{act}
	% \removeinit{in} \removeinit{in1}
	% \initwitness{cs}{cs = \emptyset}
	% \initwitness{ts}{ts = \emptyset}
	\begin{align} 
		% \invariant{m3:inv5}{ c \in \{0,1\} } \\
		% \invariant{m3:inv0}{ c = \card.cs } \\
		% \initialization{m3:in0}{ c = 0 } \\ 
		% \evbcmeq{count}{m3:act0}{c}{0} \\
		% \evbcmeq{flick}{m3:act0}{c}{1}
	\end{align}
	% \[ \variable{ n : \Int } \]
	\begin{align}
		% \invariant{m3:inv1}{ n = \qsum{p}{p \in ts}{1} } \\
		% \invariant{m3:inv2}{ \finite.ts }\\
		% \invariant{m3:inv3}{ \finite.cs } \\
		% \initialization{m3:in1}{ n = 0 }  \\
		% \invariant{m3:inv4}{ \qsum{p}{p \in ts \bunion cs}{1} = n + c }\\
		% \invariant{m3:inv6}{ ts \binter cs = \emptyset } \\
		\evbcmeq{count}{m3:act1}{n}{n+1} \\
		\evguard{count}{m3:grd0}{ c = 1 }
	\end{align}
	%!TEX root=../puzzle.tex
\begin{block}
  \item   \textbf{machine} m3
  \item   \textbf{variables}
  \begin{block}
    \item   $cs$\quad(removed)
    \item   \begin{block}
      \item   one-slot channel used to notify the counter 
    \end{block}
    \item   $ts$\quad(removed)
    \item   \begin{block}
      \item   set of process detected to have visited
    \end{block}
    \item   $b$
    \item   \begin{block}
      \item    indicates termination of the process 
    \end{block}
    \item   $c$
    \item   \begin{block}
      \item    one-bit channel used to communicate 
    \end{block}
    \item   $fs$
    \item   \begin{block}
      \item    $p \in fs$ if process $p$ has flicked the switch 
    \end{block}
    \item   $n$
    \item   $vs$
    \item   \begin{block}
      \item   set of visited processes
    \end{block}
  \end{block}
  \item   %!TEX root=../puzzle.tex
\textbf{invariants}
\begin{block}
\item[ \eqref{m3:inv0} ]{$c = \card.cs $} %
\item[ \eqref{m3:inv1} ]{$n = \qsum{p}{p \in ts}{1} $} %
\item[ \eqref{m3:inv2} ]{$\finite.ts $} %
\item[ \eqref{m3:inv5} ]{$c \in \{0,1\} $} %
\item[ \eqref{m3:inv6} ]{$ts \binter cs = \emptyset $} %
\item[ \eqref{m3:inv7} ]{$fs = ts \bunion cs $} %
\end{block}

  \item   %!TEX root=../train-station-set.tex

  \item   \textbf{events}
  \begin{block}
    \item   %!TEX root=../puzzle.tex
\noindent \ref{count}  \textbf{event}
\begin{block}
  \item   \textbf{during}
  \begin{block}
  \item[ \eqref{countsch0} ]\sout{$\neg cs = \emptyset $} %
  \end{block}
  \begin{block}
  \item[ \eqref{countsch1} ]{$c = 1$} %
  \end{block}
  \item   \textbf{when}
  \begin{block}
  \item[ \eqref{countm3:grd0} ]{$c = 1$} %
  \end{block}
  \item   \textbf{begin}
  \begin{block}
  \item[ \eqref{countact0} ]\sout{$cs \bcmeq \emptyset$} %
  \item[ \eqref{countact1} ]\sout{$ts \bcmeq ts \bunion cs $} %
  \end{block}
  \begin{block}
  \item[ \eqref{countm3:act0} ]{$c \bcmeq 0$} %
  \item[ \eqref{countm3:act1} ]{$n \bcmeq n+1$} %
  \end{block}
  \item   \textbf{end} \\
\end{block}

    \item   \noindent \ref{flick} [p] \textbf{event}
  \item   \begin{block}
    \item   stuff
  \end{block}
\begin{block}
  \item   \textbf{during}
  \begin{block}
  \item[ \eqref{flickm3:csch1} ]$c = 0 $ %
  \item[ \eqref{flickm3:csch2} ]$\neg p \in fs $ %
  \item[ \eqref{flicksch2} ]$p \in vs$ %
  \end{block}
  \item   \textbf{when}
  \begin{block}
  \item[ \eqref{flickgrd1} ]$p \in vs$ %
  \item[ \eqref{flickm3:grd1} ]$c = 0 $ %
  \item[ \eqref{flickm3:grd2} ]$\neg p \in fs $ %
  \end{block}
  \item   \textbf{begin}
  \begin{block}
  \item[ \eqref{flickm3:act0} ]$c \bcmeq 1$ %
  \item[ \eqref{flickm3:act2} ]$fs \bcmeq fs \bunion \{ p \} $ %
  \end{block}
  \item   \textbf{end} \\
\end{block}

    \item   %!TEX root=../puzzle.tex
\noindent \ref{term}  \textbf{event}
\begin{block}
  \item   \textbf{during}
  \begin{block}
  \item[ \eqref{termsch2} ]\sout{$ts = \Pcs $} %
  \end{block}
  \begin{block}
  \item[ \eqref{termsch3} ]{$n = \card.\Pcs$} %
  \end{block}
  \item   \textbf{when}
  \begin{block}
  \item[ \eqref{termgrd0} ]{$vs = \Pcs$} %
  \end{block}
  \item   \textbf{begin}
  \begin{block}
  \item[ \eqref{termact0} ]{$b \bcmeq \true$} %
  \end{block}
  \item   \textbf{end} \\
\end{block}

    \item   %!TEX root=../puzzle.tex
\noindent \ref{visit} [p] \textbf{event}
\begin{block}
  \item   \textbf{begin}
  \begin{block}
  \item[ \eqref{visitact1} ]{$vs \bcmeq vs \bunion \{ p \} $} %
  \end{block}
  \item   \textbf{end} \\
\end{block}

  \end{block}
  \item   \textbf{end} \\
\end{block}

\end{machine}

\begin{machine}{m4}
	% \refines{m3}
	% \invariant{m4:inv0}{cs \subseteq cs}
\end{machine}

\end{document}
