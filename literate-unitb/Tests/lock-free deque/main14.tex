\documentclass[12pt]{amsart}
\usepackage[margin=0.5in]{geometry} 
  % see geometry.pdf on how to lay out the page. There's lots.
\usepackage{bsymb}
\usepackage{calculational}
\usepackage{ulem}
\usepackage{hyperref}
\usepackage{unitb}

\newcommand{\REQ}{\text{REQ}}

\begin{document}

\section{Strategy}

\begin{itemize}
  \item 
\end{itemize}
\section{Model m0 --- Requests and non-deterministic handling}
  %!TEX root=../puzzle.tex
\begin{block}
  \item   \textbf{machine} m0
  \item   \textbf{variables}
  \begin{block}
    \item   $b$
    \item   \begin{block}
      \item    indicates termination of the process 
    \end{block}
  \end{block}
  \item   %!TEX root=../train-station-set.tex
\textbf{progress}
\begin{block}
\item[ \eqref{m0:prog0} ]{$t \in in  \quad \mapsto\quad \neg t \in in $} %
\end{block}

  \item   \textbf{events}
  \begin{block}
    \item   %!TEX root=../puzzle.tex
\noindent \ref{term}  \textbf{event}
\begin{block}
  \item   \textbf{during}
  \begin{block}
  \item[ (\ref{term}/default) ]\sout{$\false$} %
  \end{block}
  \begin{block}
  \item[ \eqref{termsch0} ]{$\true$} %
  \end{block}
  \item   \textbf{begin}
  \begin{block}
  \item[ \eqref{termact0} ]{$b \bcmeq \true$} %
  \end{block}
  \item   \textbf{end} \\
\end{block}

  \end{block}
  \item   \textbf{end} \\
\end{block}

\begin{machine}{m0}
  \with{sets}
  \newset{\REQ} 
  \newevent{add}{add}
  \newevent{handle}{handle}
  \[ \variable{ req : \set [\REQ] } \]
  \begin{description}
    \comment{req}{set of pending requests}
  \end{description}
  \[ \param{add}{ r : \REQ } \]
  \[ \param{handle}{ r : \REQ } \]
  \begin{align*}
      \initialization{m0:init0}{ req = \emptyset } \\
      \evguard{add}{m0:guard}{ \neg r \in req } \\
      \evbcmeq{add}{m0:act0}{req}{ req \bunion \{ r \} } \\
      \cschedule{handle}{m0:sch0}{ \neg req = \emptyset } \\
      \evguard{handle}{m0:grd0}{ r \in req } \\
      \evbcmeq{handle}{m0:act0}{req}{ req \setminus \{ r \} } 
  \end{align*}

\noindent
\end{machine}
\section{Model m1 --- Version numbers and individual fairness}
  %!TEX root=../main8.tex
\begin{block}
  \item   \textbf{machine} m1
  \item   \textbf{variables}
  \begin{block}
    \item   $emp$
    \item   $p$
    \item   $ppL$
    \item   $ppR$
    \item   $psL$
    \item   $psR$
    \item   $q$
    \item   $qe$
    \item   $res$
  \end{block}
  \item   %!TEX root=../main.tex
\textbf{invariants}
\begin{block}
\item[ \eqref{m1:inv0} ]{$\between{0}{c}{v} $} %
\item[ \eqref{m1:inv1} ]{$c > 0 \land in.c \in \dom.parse  %
            \2\implies ast = parse.(in.c) $} %
\item[ \eqref{m1:inv2} ]{$err \2\equiv c > 0  %
            \1\land \neg in.c \in \dom.parse $} %
\item[ \eqref{m1:inv3} ]{$\between{c}{nx}{v} $} %
\end{block}

  \item   \textbf{events}
  \begin{block}
    \item   %!TEX root=../main8.tex
\noindent \ref{req:pop:left}  \textbf{event}
\begin{block}
  \item   \textbf{during}
  \begin{block}
  \item[ (\ref{req:pop:left}/default) ]{$\false $} %
  \end{block}
  \item   \textbf{any} r
  \item   \textbf{when}
  \begin{block}
  \item[ \eqref{req:pop:leftm0:grd0} ]{$\neg r \in ppL $} %
  \end{block}
  \item   \textbf{begin}
  \begin{block}
  \item[ \eqref{req:pop:leftm0:act0} ]{$ppL \bcmeq ppL \1\bunion \{r\} $} %
  \end{block}
  \item   \textbf{end} \\
\end{block}

    \item   %!TEX root=../main8.tex
\noindent \ref{req:pop:right}  \textbf{event}
\begin{block}
  \item   \textbf{during}
  \begin{block}
  \item[ (\ref{req:pop:right}/default) ]{$\false $} %
  \end{block}
  \item   \textbf{any} r
  \item   \textbf{when}
  \begin{block}
  \item[ \eqref{req:pop:rightm0:grd0} ]{$\neg r \in ppR $} %
  \end{block}
  \item   \textbf{begin}
  \begin{block}
  \item[ \eqref{req:pop:rightm0:act0} ]{$ppL \bcmeq ppL \1\bunion \{r\} $} %
  \end{block}
  \item   \textbf{end} \\
\end{block}

    \item   %!TEX root=../main8.tex
\noindent \ref{req:push:left}  \textbf{event}
\begin{block}
  \item   \textbf{during}
  \begin{block}
  \item[ (\ref{req:push:left}/default) ]{$\false $} %
  \end{block}
  \item   \textbf{any} r,x
  \item   \textbf{when}
  \begin{block}
  \item[ \eqref{req:push:leftm0:grd0} ]{$\neg r \in \dom.psL $} %
  \end{block}
  \item   \textbf{begin}
  \begin{block}
  \item[ \eqref{req:push:leftm0:act0} ]{$psL \bcmeq psL \1| r \fun x $} %
  \end{block}
  \item   \textbf{end} \\
\end{block}

    \item   %!TEX root=../main8.tex
\noindent \ref{req:push:right}  \textbf{event}
\begin{block}
  \item   \textbf{during}
  \begin{block}
  \item[ (\ref{req:push:right}/default) ]{$\false $} %
  \end{block}
  \item   \textbf{any} r,x
  \item   \textbf{when}
  \begin{block}
  \item[ \eqref{req:push:rightm0:grd0} ]{$\neg r \in \dom.psR $} %
  \end{block}
  \item   \textbf{begin}
  \begin{block}
  \item[ \eqref{req:push:rightm0:act0} ]{$psR \bcmeq psR \1| r \fun x $} %
  \end{block}
  \item   \textbf{end} \\
\end{block}

    \item   %!TEX root=../main8.tex
\noindent \ref{resp:pop:left} [r] \textbf{event}
\begin{block}
  \item   \textbf{during}
  \begin{block}
  \item[ \eqref{resp:pop:leftm0:sch0} ]{$r \in ppL $} %
  \end{block}
  \item   \textbf{upon}
  \begin{block}
  \item[ \eqref{resp:pop:leftm1:sch0} ]{$\neg p = q$} %
  \end{block}
  \item   \textbf{when}
  \begin{block}
  \item[ \eqref{resp:pop:leftm1:grd0} ]{$\neg p = q$} %
  \end{block}
  \item   \textbf{begin}
  \begin{block}
  \item[ \eqref{resp:pop:leftm0:act0} ]{$ppL \bcmeq ppL \setminus \{ r \} $} %
  \item[ \eqref{resp:pop:leftm1:act0} ]{$p \bcmeq p+1$} %
  \item[ \eqref{resp:pop:leftm1:act1} ]{$qe \bcmeq \{ p \} \domsub qe $} %
  \item[ \eqref{resp:pop:leftm1:act2} ]{$emp \bcmeq \false $} %
  \item[ \eqref{resp:pop:leftm1:act3} ]{$res \bcmeq qe.p $} %
  \end{block}
  \item   \textbf{end} \\
\end{block}

    \item   %!TEX root=../main8.tex
\noindent \ref{resp:pop:left:empty} [r] \textbf{event}
\begin{block}
  \item   \textbf{during}
  \begin{block}
  \item[ \eqref{resp:pop:left:emptym0:sch0} ]{$r \in ppL $} %
  \end{block}
  \item   \textbf{begin}
  \begin{block}
  \item[ \eqref{resp:pop:left:emptym0:act0} ]{$ppL \bcmeq ppL \setminus \{ r \} $} %
  \end{block}
  \item   \textbf{end} \\
\end{block}
  \item   \textbf{upon}
\begin{block}
  \begin{block}
  \item[ \eqref{resp:pop:left:emptym1:sch0} ]{$p = q $} %
  \end{block}
  \item   \textbf{when}
  \begin{block}
  \item[ \eqref{resp:pop:left:emptym1:grd0} ]{$p = q $} %
  \end{block}
  \item   \textbf{begin}
  \begin{block}
  \item[ \eqref{resp:pop:left:emptym0:act0} ]{$ppL \bcmeq ppL \setminus \{ r \} $} %
  \item[ \eqref{resp:pop:left:emptym1:act2} ]{$emp \bcmeq \true $} %
  \end{block}
  \item   \textbf{end} \\
\end{block}

    \item   %!TEX root=../main8.tex
\noindent \ref{resp:pop:right} [r] \textbf{event}
\begin{block}
  \item   \textbf{during}
  \begin{block}
  \item[ \eqref{resp:pop:rightm0:sch0} ]{$r \in ppR $} %
  \end{block}
  \item   \textbf{upon}
  \begin{block}
  \item[ \eqref{resp:pop:rightm1:sch0} ]{$\neg p = q$} %
  \end{block}
  \item   \textbf{when}
  \begin{block}
  \item[ \eqref{resp:pop:rightm1:grd0} ]{$\neg p = q$} %
  \end{block}
  \item   \textbf{begin}
  \begin{block}
  \item[ \eqref{resp:pop:rightm0:act0} ]{$ppR \bcmeq ppR \setminus \{ r \} $} %
  \item[ \eqref{resp:pop:rightm1:act0} ]{$q \bcmeq q-1$} %
  \item[ \eqref{resp:pop:rightm1:act1} ]{$qe \bcmeq \{ q\0-1 \} \domsub qe $} %
  \item[ \eqref{resp:pop:rightm1:act2} ]{$emp \bcmeq \false $} %
  \item[ \eqref{resp:pop:rightm1:act3} ]{$res \bcmeq qe.(q\0-1) $} %
  \end{block}
  \item   \textbf{end} \\
\end{block}

    \item   %!TEX root=../main8.tex
\noindent \ref{resp:pop:right:empty} [r] \textbf{event}
\begin{block}
  \item   \textbf{during}
  \begin{block}
  \item[ \eqref{resp:pop:right:emptym0:sch0} ]{$r \in ppR $} %
  \end{block}
  \item   \textbf{upon}
  \begin{block}
  \item[ \eqref{resp:pop:right:emptym1:sch0} ]{$p = q $} %
  \end{block}
  \item   \textbf{when}
  \begin{block}
  \item[ \eqref{resp:pop:right:emptym1:grd0} ]{$p = q $} %
  \end{block}
  \item   \textbf{begin}
  \begin{block}
  \item[ \eqref{resp:pop:right:emptym0:act0} ]{$ppR \bcmeq ppR \setminus \{ r \} $} %
  \item[ \eqref{resp:pop:right:emptym1:act2} ]{$emp \bcmeq \true $} %
  \end{block}
  \item   \textbf{end} \\
\end{block}

    \item   %!TEX root=../main8.tex
\noindent \ref{resp:push:left} [r] \textbf{event}
\begin{block}
  \item   \textbf{during}
  \begin{block}
  \item[ \eqref{resp:push:leftm0:sch0} ]{$r \in \dom.psL $} %
  \end{block}
  \item   \textbf{when}
  \begin{block}
  \item[ \eqref{resp:push:leftm1:grd0} ]{$r \in \dom.psL $} %
  \end{block}
  \item   \textbf{begin}
  \begin{block}
  \item[ \eqref{resp:push:leftm0:act0} ]{$psL \bcmeq \{r\} \domsub psL $} %
  \item[ \eqref{resp:push:leftm1:act0} ]{$p \bcmeq p-1$} %
  \item[ \eqref{resp:push:leftm1:act1} ]{$qe \bcmeq qe \1| p\0-1 \fun psL.r$} %
  \end{block}
  \item   \textbf{end} \\
\end{block}

    \item   %!TEX root=../main8.tex
\noindent \ref{resp:push:right} [r] \textbf{event}
\begin{block}
  \item   \textbf{during}
  \begin{block}
  \item[ \eqref{resp:push:rightm0:sch0} ]{$r \in \dom.psR $} %
  \end{block}
  \item   \textbf{when}
  \begin{block}
  \item[ \eqref{resp:push:rightm1:grd0} ]{$r \in \dom.psR $} %
  \end{block}
  \item   \textbf{begin}
  \begin{block}
  \item[ \eqref{resp:push:rightm0:act0} ]{$psR \bcmeq \{r\} \domsub psR $} %
  \item[ \eqref{resp:push:rightm1:act0} ]{$q \bcmeq q+1$} %
  \item[ \eqref{resp:push:rightm1:act1} ]{$qe \bcmeq qe \1| q \fun psR.r $} %
  \end{block}
  \item   \textbf{end} \\
\end{block}

  \end{block}
  \item   \textbf{end} \\
\end{block}

\begin{machine}{m1}
    \refines{m0}
  \[ \variable{ ver : \Int } \]
  \begin{description}
    \comment{ver}{ serial number of the current data structure state }
  \end{description}
  \[ \indices{handle}{ v : \Int } \]
  \promote{handle}{r}
  \removecoarse{handle}{m0:sch0}
  \removeguard{handle}{m0:grd0}
  % \removeact{handle}{m0:act0}
  \[\witness{handle}{r}{r \in req}\]
  \[\witness{handle}{v}{v = ver}\]
  \begin{align}
      \initialization{m1:init0}{ ver = 0 } \\
      \cschedule{handle}{m1:sch0}{ r \in req } \\
      \cschedule{handle}{m1:sch1}{ v = ver } \\
      % \evguard{handle}{m1:grd0}{ r = r0 } \\
      \evbcmeq{handle}{m1:act0}{ver}{ ver + 1 } 
  \end{align}
\end{machine}
\section{Model m2 --- Specialized events}
  %!TEX root=../main9.tex
\begin{block}
  \item   \textbf{machine} m2
  \item   \textbf{refines} m0
  \item   \textbf{variables}
  \begin{block}
    \item   $req$
    \item   $req0$
    \item   $reqA$
    \item   $reqB$
  \end{block}
  \item   %!TEX root=../main.tex
\textbf{invariants}
\begin{block}
\item[ \eqref{m2:inv0} ]{$r \in reach $} %
\end{block}

  \item   \textbf{events}
  \begin{block}
    \item   %!TEX root=../main9.tex
\noindent \ref{handle}  \textbf{event}
\begin{block}
  \item   \textbf{during}
  \begin{block}
  \item[ \eqref{handlem0:sch0} ]{$\neg req = \emptyset$} %
  \end{block}
  \item   \textbf{any} r
  \item   \textbf{when}
  \begin{block}
  \item[ \eqref{handlegrd0} ]{$r \in req$} %
  \end{block}
  \item   \textbf{begin}
  \begin{block}
  \item[ \eqref{handleact0} ]{$req \bcmeq req \setminus \{ r \}$} %
  \item[ \eqref{handleact1} ]{$req0 \bcmeq req$} %
  \end{block}
  \item   \textbf{end} \\
\end{block}

    \item   %!TEX root=../main9.tex
\noindent \ref{req}  \textbf{event}
\begin{block}
  \item   \textbf{during}
  \begin{block}
  \item[ (\ref{req}/default) ]{$\false$} %
  \end{block}
  \item   \textbf{any} r
  \item   \textbf{when}
  \begin{block}
  \item[ \eqref{reqgrd0} ]{$\neg r \in req$} %
  \end{block}
  \item   \textbf{begin}
  \begin{block}
  \item[ \eqref{reqact0} ]{$req \bcmeq req \bunion \{ r \}$} %
  \item[ \eqref{reqact1} ]{$req0 \bcmeq req$} %
  \end{block}
  \item   \textbf{end} \\
\end{block}

  \end{block}
  \item   \textbf{end} \\
\end{block}

\begin{machine}{m2}
  \refines{m1}
  \[ \variable{pshL,pshR,popR,popL : \set [\REQ]} \]
  \begin{description}
    \comment{pshL}{replaces $req$. Set of push\_left requests }
    \comment{pshR}{replaces $req$. Set of push\_right requests }
    \comment{popL}{replaces $req$. Set of pop\_left requests }
    \comment{popR}{replaces $req$. Set of pop\_right requests }
  \end{description}
  \begin{align}
    \initialization{m2:init0}{pshL = \emptyset} \\
    \initialization{m2:init1}{popL = \emptyset} \\
    \initialization{m2:init2}{pshR = \emptyset} \\
    \initialization{m2:init3}{popR = \emptyset} 
  \end{align}
  \subsection{Specialize \emph{handle}}
  % \begin{align*}
    \refiningevent{handle}{handle:popL}{handle\_popL}
    \refiningevent{handle}{handle:popR}{handle\_popR} 
    \refiningevent{handle}{handle:pushL}{handle\_pushL}
    \refiningevent{handle}{handle:pushR}{handle\_pushR}
  % \end{align*}
  \splitevent{handle}{handle:popL,handle:popR,handle:pushR,handle:pushL}
  \begin{align*}
    \definition{Req}{ pshL \bunion 
        pshR \bunion 
        popL \bunion 
        popR } 
  \end{align*}
  \begin{align}
    \invariant{m2:inv0}{ pshL \bunion 
        pshR \bunion 
        popL \bunion 
        popR 
        = req } \\
    \invariant{m2:inv1}{ pshL \binter pshR = \emptyset } \\
    \invariant{m2:inv2}{ pshL \binter popL = \emptyset } \\
    \invariant{m2:inv3}{ pshL \binter popR = \emptyset } \\
    \invariant{m2:inv4}{ pshR \binter popL = \emptyset } \\
    \invariant{m2:inv5}{ pshR \binter popR = \emptyset } \\
    \invariant{m2:inv6}{ popL \binter popR = \emptyset } \\
    \cschedule{handle:pushL}{m2:sch0}{ r \in pshL } \\
    \evbcmeq{handle:pushL}{m2:act0}{pshL}{pshL \setminus \{ r \}} \\
    \cschedule{handle:pushR}{m2:sch0}{ r \in pshR } \\
    \evbcmeq{handle:pushR}{m2:act0}{pshR}{pshR \setminus \{ r \}} \\
    \cschedule{handle:popL}{m2:sch0}{ r \in popL } \\
    \evbcmeq{handle:popL}{m2:act0}{popL}{popL \setminus \{ r \}} \\
    \cschedule{handle:popR}{m2:sch0}{ r \in popR } \\
    \evbcmeq{handle:popR}{m2:act0}{popR}{popR \setminus \{ r \}} 
    % \invariant{m2:inv1}{ popL \subseteq req } \\
    % \invariant{m2:inv2}{ pshR \subseteq req } \\
    % \invariant{m2:inv3}{ popR \subseteq req } 
  \end{align}
  \[ \variable{ ppd : \set [\REQ] } \]
  \begin{align}
    \initialization{m2:init4}{ ppd = \emptyset } \\
    \evbcmeq{handle:popR}{m2:act1}{ppd}{ ppd \bunion \{r\} } \\
    \evbcmeq{handle:popL}{m2:act1}{ppd}{ ppd \bunion \{r\} }
  \end{align}
  \newevent{return}{return}
  \[ \indices{return}{ r : \REQ } \]
  \begin{align}
    \cschedule{return}{m2:sch0}{ r \in ppd } \\
    \evbcmeq{return}{m2:act0}{ppd}{ ppd \setminus \{ r \} }
  \end{align}
\subsection{Specialize \emph{add}}
    \refiningevent{add}{add:popL}{add\_popL} 
    \refiningevent{add}{add:popR}{add\_popR}
    \refiningevent{add}{add:pushL}{add\_pushL}
    \refiningevent{add}{add:pushR}{add\_pushR}
  \hide{\splitevent{add}{add:popL,add:popR,add:pushR,add:pushL}}
  \begin{align}
    \evbcmeq{add:popL}{m2:act0}{popL}{popL \bunion \{r\}} \\
    \evbcmeq{add:pushL}{m2:act1}{pshL}{pshL \bunion \{r\}} \\
    \evbcmeq{add:popR}{m2:act2}{popR}{popR \bunion \{r\}} \\
    \evbcmeq{add:pushR}{m2:act3}{pshR}{pshR \bunion \{r\}} 
  \end{align}
\subsection{Data refinement}
  \removevar{req}
  \removeinit{m0:init0}
  \[\initwitness{req}{req = \emptyset}\]
  \removeact{handle:popR}{m0:act0}
  \removeact{handle:pushR}{m0:act0}
  \removeact{handle:popL}{m0:act0}
  \removeact{handle:pushL}{m0:act0}
  \removecoarse{handle:popR}{m1:sch0}
  \removecoarse{handle:pushR}{m1:sch0}
  \removecoarse{handle:popL}{m1:sch0}
  \removecoarse{handle:pushL}{m1:sch0}
  \removeguard{add:popR}{m0:guard}
  \removeguard{add:pushR}{m0:guard}
  \removeguard{add:popL}{m0:guard}
  \removeguard{add:pushL}{m0:guard}
  \begin{align*}
  \evguard{add:popR}{m2:grd0}
    { \neg r \in Req } \\
  \evguard{add:pushR}{m2:grd0}
    { \neg r \in Req } \\
  \evguard{add:popL}{m2:grd0}
    { \neg r \in Req } \\
  \evguard{add:pushL}{m2:grd0}
    { \neg r \in pshL \bunion pshR \bunion popL \bunion popR } 
  \end{align*}
  \removeact{add:popR}{m0:act0}
  \removeact{add:pushR}{m0:act0}
  \removeact{add:popL}{m0:act0}
  \removeact{add:pushL}{m0:act0}
\end{machine}
  \newcommand{\OBJ}{\text{OBJ}}
\section{Model m3 --- The Contents}
  %!TEX root=../puzzle.tex
\begin{block}
  \item   \textbf{machine} m3
  \item   \textbf{variables}
  \begin{block}
    \item   $cs$\quad(removed)
    \item   \begin{block}
      \item   one-slot channel used to notify the counter 
    \end{block}
    \item   $ts$\quad(removed)
    \item   \begin{block}
      \item   set of process detected to have visited
    \end{block}
    \item   $b$
    \item   \begin{block}
      \item    indicates termination of the process 
    \end{block}
    \item   $c$
    \item   \begin{block}
      \item    one-bit channel used to communicate 
    \end{block}
    \item   $fs$
    \item   \begin{block}
      \item    $p \in fs$ if process $p$ has flicked the switch 
    \end{block}
    \item   $n$
    \item   $vs$
    \item   \begin{block}
      \item   set of visited processes
    \end{block}
  \end{block}
  \item   %!TEX root=../puzzle.tex
\textbf{invariants}
\begin{block}
\item[ \eqref{m3:inv0} ]{$c = \card.cs $} %
\item[ \eqref{m3:inv1} ]{$n = \qsum{p}{p \in ts}{1} $} %
\item[ \eqref{m3:inv2} ]{$\finite.ts $} %
\item[ \eqref{m3:inv5} ]{$c \in \{0,1\} $} %
\item[ \eqref{m3:inv6} ]{$ts \binter cs = \emptyset $} %
\item[ \eqref{m3:inv7} ]{$fs = ts \bunion cs $} %
\end{block}

  \item   %!TEX root=../train-station-set.tex

  \item   \textbf{events}
  \begin{block}
    \item   %!TEX root=../puzzle.tex
\noindent \ref{count}  \textbf{event}
\begin{block}
  \item   \textbf{during}
  \begin{block}
  \item[ \eqref{countsch0} ]\sout{$\neg cs = \emptyset $} %
  \end{block}
  \begin{block}
  \item[ \eqref{countsch1} ]{$c = 1$} %
  \end{block}
  \item   \textbf{when}
  \begin{block}
  \item[ \eqref{countm3:grd0} ]{$c = 1$} %
  \end{block}
  \item   \textbf{begin}
  \begin{block}
  \item[ \eqref{countact0} ]\sout{$cs \bcmeq \emptyset$} %
  \item[ \eqref{countact1} ]\sout{$ts \bcmeq ts \bunion cs $} %
  \end{block}
  \begin{block}
  \item[ \eqref{countm3:act0} ]{$c \bcmeq 0$} %
  \item[ \eqref{countm3:act1} ]{$n \bcmeq n+1$} %
  \end{block}
  \item   \textbf{end} \\
\end{block}

    \item   \noindent \ref{flick} [p] \textbf{event}
  \item   \begin{block}
    \item   stuff
  \end{block}
\begin{block}
  \item   \textbf{during}
  \begin{block}
  \item[ \eqref{flickm3:csch1} ]$c = 0 $ %
  \item[ \eqref{flickm3:csch2} ]$\neg p \in fs $ %
  \item[ \eqref{flicksch2} ]$p \in vs$ %
  \end{block}
  \item   \textbf{when}
  \begin{block}
  \item[ \eqref{flickgrd1} ]$p \in vs$ %
  \item[ \eqref{flickm3:grd1} ]$c = 0 $ %
  \item[ \eqref{flickm3:grd2} ]$\neg p \in fs $ %
  \end{block}
  \item   \textbf{begin}
  \begin{block}
  \item[ \eqref{flickm3:act0} ]$c \bcmeq 1$ %
  \item[ \eqref{flickm3:act2} ]$fs \bcmeq fs \bunion \{ p \} $ %
  \end{block}
  \item   \textbf{end} \\
\end{block}

    \item   %!TEX root=../puzzle.tex
\noindent \ref{term}  \textbf{event}
\begin{block}
  \item   \textbf{during}
  \begin{block}
  \item[ \eqref{termsch2} ]\sout{$ts = \Pcs $} %
  \end{block}
  \begin{block}
  \item[ \eqref{termsch3} ]{$n = \card.\Pcs$} %
  \end{block}
  \item   \textbf{when}
  \begin{block}
  \item[ \eqref{termgrd0} ]{$vs = \Pcs$} %
  \end{block}
  \item   \textbf{begin}
  \begin{block}
  \item[ \eqref{termact0} ]{$b \bcmeq \true$} %
  \end{block}
  \item   \textbf{end} \\
\end{block}

    \item   %!TEX root=../puzzle.tex
\noindent \ref{visit} [p] \textbf{event}
\begin{block}
  \item   \textbf{begin}
  \begin{block}
  \item[ \eqref{visitact1} ]{$vs \bcmeq vs \bunion \{ p \} $} %
  \end{block}
  \item   \textbf{end} \\
\end{block}

  \end{block}
  \item   \textbf{end} \\
\end{block}

\begin{machine}{m3}
  \refines{m2}
  \with{functions}
  \with{intervals}

  \newset{\OBJ} 
  \[ \variable{ p,q : \Int } \]
  \[ \variable{ qe : \Int \pfun \OBJ } \]
  \begin{align}
    \invariant{m3:inv0}{ qe \in \intervalR{p}{q} \tfun \OBJ } \\
    \invariant{m3:inv1}{ p \le q } \\
    \initialization{m3:init0}{ p = 0 \land q = 0} \\
    \initialization{m3:init1}{ qe = \emptyfun }
  \end{align}
  \subsection{Push}
    \[\variable{insL : \REQ \pfun \OBJ} \]
    \[\variable{insR : \REQ \pfun \OBJ} \]
  \begin{description}
    \comment{insL}{parameter for the \emph{push left} operation} 
    \comment{insR}{parameter for the \emph{push right} operation} 
  \end{description}
  \begin{align}
    \invariant{m3:inv2}{ insL \in pshL \tfun \OBJ } \\
    \invariant{m3:inv3}{ insR \in pshR \tfun \OBJ } \\
    \initialization{m3:init2}{ insL = \emptyfun } \\
    \initialization{m3:init4}{ insR = \emptyfun }
  \end{align}
  \begin{align}
    \evbcmeq{handle:pushL}{m3:act0}{p}{p - 1} \\
    \evbcmeq{handle:pushL}{m3:act1}{qe}{ qe \2| (p \0- 1 \fun insL.r)} \\
    \evbcmeq{handle:pushR}{m3:act0}{q}{q + 1} \\
    \evbcmeq{handle:pushR}{m3:act1}{qe}{ qe \2| (q \fun insR.r)} 
  \end{align}
  
  \[ \param{add:pushL}{ obj : \OBJ } \]

  \begin{align}
    \evbcmeq{add:pushL}{m3:act0}
      {insL}{ insL \1| r \fun obj } \\
    \evbcmeq{handle:pushL}{m3:act2}
      {insL}{ \{ r \} \domsub insL }
  \end{align}

  \[ \param{add:pushR}{ obj : \OBJ } \]
  
  \begin{align}
    \evbcmeq{add:pushR}{m3:act0}
      {insR}{ insR \1| r \fun obj } \\
    \evbcmeq{handle:pushR}{m3:act2}
      {insR}{ \{ r \} \domsub insR }
  \end{align}
  \[ \variable{ res : \REQ \pfun \OBJ } \]
  \[ \variable{ result : \OBJ } \]
  \[ \variable{ emp : \Bool } \]
  \subsection{Pop}
  \begin{align}
    \initialization{m3:init3}{ res = \emptyfun } \\
    \invariant{m3:inv4}{ res \in ppd \pfun \OBJ } \\
    \evbcmeq{return}{m3:act0}{res}{ \{r\} \domsub res} \\
    \evbcmsuch{return}{m3:act1}{result}{ r \in \dom.res \implies result' = res.r } \\
    \evbcmeq{return}{m3:act2}{emp}{ (r \in \dom.res) } 
  \end{align}
  \splitevent{handle:popL}{handle:popL:empty,handle:popL:non:empty}
  \splitevent{handle:popR}{handle:popR:empty,handle:popR:non:empty}
  \refiningevent{handle:popL}{handle:popL:empty}{handle\_popL\_empty}
  \refiningevent{handle:popL}{handle:popL:non:empty}{handle\_popL\_non\_empty}
  \refiningevent{handle:popR}{handle:popR:empty}{handle\_popR\_empty}
  \refiningevent{handle:popR}{handle:popR:non:empty}{handle\_popR\_non\_empty}
  \[ \dummy{v : \Int} \]
  \replace{handle:popL}{m3:sch0}{m3:prog0}
  \begin{align*}
    & \progress{m3:prog0}{v = ver}{p = q \lor p < q}
    \refine{m3:prog0}{implication}{}{}
  \end{align*}
  \begin{align}
    \cschedule{handle:popL:non:empty}{m3:sch0}{ p < q } \\
    \cschedule{handle:popL:empty}{m3:sch0}{ p = q } \\
    \evbcmeq{handle:popL:non:empty}{m3:act0}{res}{ res \1| (r \fun qe.p) } \\
    \evbcmeq{handle:popL:non:empty}{m3:act1}{p}{p+1} \\
    \evbcmeq{handle:popL:non:empty}{m3:act2}{qe}{ \{p\} \domsub qe }
  \end{align}
  \replace{handle:popR}{m3:sch0}{m3:prog0}
  \begin{align}
    \cschedule{handle:popR:non:empty}{m3:sch0}{ p < q } \\
    \cschedule{handle:popR:empty}{m3:sch0}{ p = q } \\
    \evbcmeq{handle:popR:non:empty}{m3:act0}{res}{ res \1| (r \fun qe.(q\0-1)) } \\
    \evbcmeq{handle:popR:non:empty}{m3:act1}{q}{q-1} \\
    \evbcmeq{handle:popR:non:empty}{m3:act2}{qe}{ \{q-1\} \domsub qe }
  \end{align}
\end{machine}
\section{Model m4 --- Memory nodes}
\newcommand{\Node}{\text{Node}}
  \begin{block}
  \item   \textbf{machine} m4
  \item   \textbf{variables}
  \begin{block}
    \item   $b$
    \item   \begin{block}
      \item    indicates termination of the process 
    \end{block}
    \item   $c$
    \item   \begin{block}
      \item    one-bit channel used to communicate 
    \end{block}
    \item   $fs$
    \item   \begin{block}
      \item    $p \in fs$ if process $p$ has flicked the switch 
    \end{block}
    \item   $n$
    \item   $vs$
    \item   \begin{block}
      \item   set of visited processes
    \end{block}
  \end{block}
  \item   \textbf{events}
  \begin{block}
    \item   \noindent \ref{count}  \textbf{event}
\begin{block}
  \item   \textbf{during}
  \begin{block}
  \item[ \eqref{countsch1} ]$c = 1$ %
  \end{block}
  \item   \textbf{when}
  \begin{block}
  \item[ \eqref{countm3:grd0} ]$c = 1$ %
  \end{block}
  \item   \textbf{begin}
  \begin{block}
  \item[ \eqref{countm3:act0} ]$c \bcmeq 0$ %
  \item[ \eqref{countm3:act1} ]$n \bcmeq n+1$ %
  \end{block}
  \item   \textbf{end} \\
\end{block}

    \item   \noindent \ref{flick} [p] \textbf{event}
  \item   \begin{block}
    \item   stuff
  \end{block}
\begin{block}
  \item   \textbf{during}
  \begin{block}
  \item[ \eqref{flickm3:csch1} ]$c = 0 $ %
  \item[ \eqref{flickm3:csch2} ]$\neg p \in fs $ %
  \item[ \eqref{flicksch2} ]$p \in vs$ %
  \end{block}
  \item   \textbf{when}
  \begin{block}
  \item[ \eqref{flickgrd1} ]$p \in vs$ %
  \item[ \eqref{flickm3:grd1} ]$c = 0 $ %
  \item[ \eqref{flickm3:grd2} ]$\neg p \in fs $ %
  \end{block}
  \item   \textbf{begin}
  \begin{block}
  \item[ \eqref{flickm3:act0} ]$c \bcmeq 1$ %
  \item[ \eqref{flickm3:act2} ]$fs \bcmeq fs \bunion \{ p \} $ %
  \end{block}
  \item   \textbf{end} \\
\end{block}

    \item   %!TEX root=../puzzle.tex
\noindent \ref{term}  \textbf{event}
\begin{block}
  \item   \textbf{during}
  \begin{block}
  \item[ \eqref{termsch3} ]{$n = \card.\Pcs$} %
  \end{block}
  \item   \textbf{when}
  \begin{block}
  \item[ \eqref{termgrd0} ]{$vs = \Pcs$} %
  \end{block}
  \item   \textbf{begin}
  \begin{block}
  \item[ \eqref{termact0} ]{$b \bcmeq \true$} %
  \end{block}
  \item   \textbf{end} \\
\end{block}

    \item   %!TEX root=../puzzle.tex
\noindent \ref{visit} [p] \textbf{event}
\begin{block}
  \item   \textbf{begin}
  \begin{block}
  \item[ \eqref{visitact1} ]{$vs \bcmeq vs \bunion \{ p \} $} %
  \end{block}
  \item   \textbf{end} \\
\end{block}

  \end{block}
  \item   \textbf{end} \\
\end{block}

\begin{machine}{m4}
  \refines{m3}
\subsection{Data structure}
  \newset{\Node}
  \[ \definition{Node}{ 
      \begin{array}{l}
      [ 'item : \OBJ ] 
      \end{array}
      } \]
  \[ \variable{ rep : \Int \pfun \Node } \]
  \[ \variable{ item : \Node \pfun \OBJ } \]
  \[ \variable{ node : \set [\Node] } \]
  \[ \constant{ dummy : \Node } \]
  \begin{align}
    \invariant{m4:inv0}
      { rep \in \intervalR{p}{q} \tfun node } \\
    \invariant{m4:inv1}
      { item \in node \tfun \OBJ } \\
    % \invariant{m4:inv2}{ rep is injective }
    \invariant{m4:inv3}
      { \qforall{i}{\betweenR{p}{i}{q}}{ qe.i = item.(rep.i) } } \\
    \initialization{m4:init0}{ rep  = \emptyfun } \\
    \initialization{m4:init1}
      { \qexists{val}{}{ item \1= dummy \fun val } } \\
    \initialization{m4:init2}{ node = \{ dummy \} } \\
    \invariant{m4:inv9}
      { dummy \in node } \\
    \invariant{m4:inv10}
      { \neg dummy \in \ran.rep }
  \end{align}
\subsection{Pop}
  \begin{align}
    \evbcmeq{handle:popL:non:empty}{m4:act0}{rep}
      { \{ p \} \domsub rep } \\
    \evbcmeq{handle:popR:non:empty}{m4:act0}{rep}
      { \{ q-1 \} \domsub rep }
  \end{align}
\subsection{Push}
  \[ \variable{ nL : \REQ \pfun \Node } \]
  \[ \variable{ nR : \REQ \pfun \Node } \]
  \[ \variable{ new : \set [\Node] } \]
\paragraph{\eqref{m4:inv0}}
  \begin{align}
    \invariant{m4:inv4}{ nL \in pshL \pfun node } \\
    \invariant{m4:inv5}{ nR \in pshR \pfun node } \\
    \initialization{m4:init3}{ nL = \emptyfun } \\
    \initialization{m4:init4}{ nR = \emptyfun } \\
    \evbcmeq{handle:pushL}{m4:act0}
      {rep}{ rep \2| (p\0-1 \fun nL.r) } \\
    \cschedule{handle:pushL}{m4:sch0}
      { r \in \dom.nL } \\
    \evbcmeq{handle:pushR}{m4:act0}
      {rep}{ rep \2| (q \fun nR.r) } \\
    \cschedule{handle:pushR}{m4:sch0}
      { r \in \dom.nR } 
  \end{align}
  \replace{handle:pushL}{m4:sch0}{m4:prog0}
  \replace{handle:pushR}{m4:sch0}{m4:prog1}
  \[ \dummy{ r : \REQ } \]
  \begin{align}
    \progress{m4:prog0}{ r \in pshL }{ r \in \dom.nL } \\
    \progress{m4:prog1}{ r \in pshR }{ r \in \dom.nR }
  \end{align}
\paragraph{\eqref{m4:inv3}}
  \begin{align}
    \invariant{m4:inv6}
      { \qforall{r}{ r \in \dom.nL }{item.(nL.r) = insL.r} }\\
    \evbcmeq{handle:pushL}{m4:act1}{nL}{ \{ r \} \domsub nL }
  \end{align}  
  \begin{align}
    \invariant{m4:inv7}
      { \qforall{r}{ r \in \dom.nR }{item.(nR.r) = insR.r} }\\
    \evbcmeq{handle:pushR}{m4:act1}{nR}{ \{ r \} \domsub nR }
  \end{align} 
\subsection{ \eqref{m4:inv10} } 
  \begin{align}
    \invariant{m4:inv11}
      { \neg dummy \in \ran.nL } \\
    \invariant{m4:inv12}
      { \neg dummy \in \ran.nR }
  \end{align}
\subsection{New Progress Properties}
  \newevent{allocateL}{allocate\_left}
  \begin{align*}
    \refine{m4:prog0}{ensure}{allocateL}{ \index{r}{r' = r} }
  \end{align*}
  \[ \indices{allocateL}{ r : \REQ } \]
  \[ \param{allocateL}{ n : \Node } \]
  \begin{align}
    \evguard{allocateL}{m4:grd0}{ \neg n \in node } \\
    \assumption{m4:asm0}{ \neg \finite{\Node} } \\
    \invariant{m4:inv8}{ \finite{node} } \\
    \cschedule{allocateL}{m4:sch0}{ r \in pshL } \\
    \evbcmeq{allocateL}{m4:act0}{nL}{nL \1| (r \fun n)} \\
    \evbcmeq{allocateL}{m4:act1}{item}{item \1| (n \fun insL.r)} \\
    \evbcmeq{allocateL}{m4:act2}{node}{ node \bunion \{n\}}
  \end{align}
  \newevent{allocateR}{allocate\_right}
  \begin{align*}
    \refine{m4:prog1}{ensure}{allocateR}{ \index{r}{r' = r} }
  \end{align*}
  \[ \indices{allocateR}{ r : \REQ } \]
  \[ \param{allocateR}{ n : \Node } \]
  \begin{align}
    \evguard{allocateR}{m4:grd0}{ \neg n \in node } \\
    \cschedule{allocateR}{m4:sch0}{ r \in pshR } \\
    \evbcmeq{allocateR}{m4:act0}{nR}{nR \1| (r \fun n)} \\
    \evbcmeq{allocateR}{m4:act1}{item}{item \1| (n \fun insR.r)} \\
    \evbcmeq{allocateR}{m4:act2}{node}{ node \bunion \{n\}}
  \end{align}

\end{machine}
\input{lock-free-deque/machine_m5}
\begin{machine}{m5}
  \refines{m4}
  \[ \variable{ left, right : \Node \pfun \Node } \]
  \[ \variable{ LH, RH : \Node } \]
  \begin{align}
    \invariant{m5:inv0}{ left \in node \tfun node } \\
    \invariant{m5:inv1}{ right \in node \tfun node }  \\
    \invariant{m5:inv2}
      { p < q \1\implies LH = rep.p \land RH = rep.(q\0-1) }  \\
    \invariant{m5:inv3}
      { p = q \1\implies LH = dummy \land RH = dummy }  \\
    \invariant{m5:inv4}{ left.dummy = dummy }  \\
    \invariant{m5:inv5}{ right.dummy = dummy }  \\
    \initialization{m5:init0}{ left \1= (dummy \fun dummy) } \\
    \initialization{m5:init1}{ right \1= (dummy \fun dummy) } \\
    \initialization{m5:init2}{ LH = dummy \land RH = dummy }
  \end{align}

  \begin{align}
    \invariant{m5:inv6}
      { \qforall{i}{\betweenR{p}{i}{q\0-1}}{ right.(rep.i) = rep.(i+1) } } \\
    \invariant{m5:inv7}
      { \qforall{i}{\betweenR{p}{i}{q\0-1}}{ left.(rep.(i+1)) = rep.i } } 
  \end{align}
\subsection{Pop}
  % \begin{align}
  %   \invariant{m5:inv8}
  %     { left.(left.LH) = Dummy } \\
  %   \invariant{m5:inv9}
  %     { right.RH = dummy }  \\
  %   \evbcmeq{handle:popL:non:empty}{m5:act0}{LH}{right.LH}
  % \end{align}
\subsection{Push}
    \splitevent{handle:pushL}{handle:pushL:empty,handle:pushL:non:empty} 
    \refiningevent{handle:pushL}{handle:pushL:empty}
      {push\_left\_empty} 
    \refiningevent{handle:pushL}{handle:pushL:non:empty}
      {push\_left\_non\_empty}
  \begin{align} 
    \invariant{m5:inv8}{ \neg dummy \in \ran.nL } \\
    \invariant{m5:inv9}{ \neg dummy \in \ran.nR } \\
    \invariant{m5:inv10}{ \injective{rep} } \\
    \cschedule{handle:pushL:non:empty}{m5:sch0}
      { p < q } \\
    \cschedule{handle:pushL:non:empty}{m5:sch1}
      { right.(nL.r) = LH } \\
    \evbcmeq{handle:pushL:non:empty}{m5:act0}
      {left}{left \1| LH \fun nL.r} \\
    \evbcmeq{handle:pushL:non:empty}{m5:act1}
      {LH}{nL.r} \\
    \cschedule{handle:pushL:empty}{m5:sch0}
      { p = q } \\
    \evbcmeq{handle:pushL:empty}{m5:act0}
      {LH}{nL.r} \\
    \evbcmeq{handle:pushL:empty}{m5:act1}
      {RH}{nL.r}
  \end{align}
\subsection{Allocate}
  \begin{align}
    \evbcmeq{allocateR}{m5:act0}{left}{left \1| n \fun dummy} \\
    \evbcmeq{allocateR}{m5:act1}{right}{right \1| n \fun dummy} \\
    \evbcmeq{allocateL}{m5:act0}{left}{left \1| n \fun dummy} \\
    \evbcmeq{allocateL}{m5:act1}{right}{right \1| n \fun dummy} 
  \end{align}
\end{machine}
\end{document}
