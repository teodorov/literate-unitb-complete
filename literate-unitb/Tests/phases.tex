
\documentclass[12pt]{amsart}
\usepackage{geometry} % see geometry.pdf on how to lay out the page. There's lots.
\usepackage{bsymb}
\usepackage{unitb}
\usepackage{calculational}
\usepackage{ulem}
\usepackage{hyperref}
\normalem
\geometry{a4paper} % or letter or a5paper or ... etc
% \geometry{landscape} % rotated page geometry

% See the ``Article customise'' template for some common customisations

\title{}
\author{}
\date{} % delete this line to display the current date

%%% BEGIN DOCUMENT
\begin{document}

\maketitle
%\tableofcontents

%\section{}
%\subsection{}

\begin{machine}{m0}

\newset{\Pcs}

%\hide{
	\begin{align*}
\variable{		pc : \Pcs \pfun \Int}
	\end{align*}
%}

\begin{align*}
\dummy{	p,p_0,p_1: \Pcs}
\end{align*}

\begin{align*}
\dummy{	N,k: \Int}
\end{align*}

\with{functions}
\with{sets}  

\newevent{step}
\begin{align*}
&	\progress{prog0}{\true}{N \le pc.p}
\refine{prog0}{disjunction}{prog1,prog2}{ with }
&	\progress{prog2}{pc.p \le N}{N \le pc.p}
\\&	\progress{prog1}{N \le pc.p}{N \le pc.p}
\refine{prog1}{implication}{}{}
&	pc.p \le N \3\mapsto N \le pc.p \tag{\ref{prog2}}
\refine{prog2}{induction}{prog3}{ over \var{pc.p}{up}{N} }
&	\progress{prog3}{pc.p = k \land pc.p \le N}{(k < pc.p \land pc.p \le N) \lor N \le pc.p}
\refine{prog3}{PSP}{prog4,saf0}{}
&	\safety{saf0}{k \le pc.p \land pc.p \le N}{N \le pc.p}
\\ &	\progress{prog4}{pc.p = k}{\neg k = pc.p}
\refine{prog4}{discharge}{tr0}{}
&	\transient{step}{tr0}{ pc.p = k }\textbf{ is transient}
\end{align*}

\begin{align*}
\cschedule{step}{c0}{\true} \\
\evassignment{step}{a0}{ pc' \2= pc \2| p \fun (pc.p + 1) }
\end{align*}

\removecoarse{step}{c0} % \weakento{step}{c0}

\begin{align*}
\indices{step}{	p: \Pcs}
\end{align*}

%\begin{proof}{step/CO/saf0}
%	\begin{free:var}{k}{k}
%	\begin{free:var}{p}{p_0}
%	\begin{by:cases}
%		\begin{case}{hyp0}{p_0 = p}
%		\easy
%		\end{case}
%		\begin{case}{hyp1}{\neg p_0 = p}
%		\begin{calculation}
%			k \le pc'.p_0
%		\hint{=}{  \ref{a0} }
%			k \le (pc \1 | p \1\fun pc.p \0+ 1).p_0
%		\hint{=}{  \eqref{hyp1} }
%			k \le pc.p_0
%		\end{calculation}
%		\end{case}
%	\end{by:cases}
%	\end{free:var}
%	\end{free:var}
%\end{proof}

\begin{proof}{step/TR/tr0}
%	\begin{free:var}{k}{k}
%	\begin{free:var}{p}{p_0}
\begin{align}
	\assume{hyp1}{ pc' &\1= pc \1| p \fun pc.p\0+1 } \\
	\goal{\neg ~ pc'.p  &= pc.p}
%	\goal{
\end{align}
	\easy
%	\begin{subproof}{hyp0}
%	\begin{calculation}
%		pc'.p = pc.p
%	\hint{=}{ \ref{hyp1} }
%		(pc \1| p \fun pc.p + 1).p = pc.p 
%	\hint{=}{ function application }
%		pc.p + 1 = pc.p
%	\hint{=}{ function application }
%		\false
%	\end{calculation}
%	\end{subproof}
%	\begin{by:cases}
%		\begin{case}{hyp0}{p_0 = p}
%		\easy
%		\end{case}
%		\begin{case}{hyp1}{\neg p_0 = p}
%		\begin{calculation}
%			k \le pc'.p_0
%		\hint{=}{  \ref{a0} }
%			k \le (pc \1 | p \1\fun pc.p \0+ 1).p_0
%		\hint{=}{  \eqref{hyp1} }
%			k \le pc.p_0
%		\end{calculation}
%		\end{case}
%	\end{by:cases}
%	\end{free:var}
%	\end{free:var}
\end{proof}

\end{machine}

\begin{machine}{m1}

\refines{m0}

\begin{align*}
\invariant{inv0}
{	&\qforall{p_0,p_1}{}{pc.p_0 \le pc.p_1 + 1} } \\
\initialization{init0}
{	&\quad pc = \qfun{p}{}{0} }
\end{align*}

\begin{proof}{step/INV/inv0}
	\begin{free:var}{p_0}{p_0}
	\begin{free:var}{p_1}{p_1}
	\begin{by:cases}
		\begin{case}{hyp0}{p = p_1}
		easy ...
		\easy
		
		\end{case}
		\begin{case}{hyp1}{\neg ~ p = p_1 \land p = p_0}
\begin{calculation}
		pc'.p_0 \le pc'.p_1 + 1
	\hint{=}{ \ref{a0} and \eqref{hyp1} }
		pc.p + 1 \le pc.p_1 + 1
	\hint{=}{ arithmetic }
		pc.p \le pc.p_1
	\hint{=}{ \ref{grd0} }
		\true
\end{calculation}
		\end{case}
		\begin{case}{hyp2}{\neg ~ p = p_1 \1\land \neg~p = p_0}
		\easy
		\end{case}
	\end{by:cases}
	\end{free:var}
	\end{free:var}
\end{proof}

\begin{align*}
\evguard{step}{grd0}
{	\qforall{p_0}{}{ pc.p \le pc.p_0 } }
\end{align*}

\end{machine}

\end{document}